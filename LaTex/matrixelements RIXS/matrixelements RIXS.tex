\documentclass[twocolumn,prb,twocolumn,amsmath,superscriptaddress,nofootinbib,amssymb]{revtex4-1}

\usepackage{ulem}
\usepackage[usenames]{color}

\usepackage{graphicx}
\usepackage{times}
\usepackage{dcolumn}
\usepackage{slashbox,pict2e}

\def\Xint#1{\mathchoice
   {\XXint\displaystyle\textstyle{#1}}%
   {\XXint\textstyle\scriptstyle{#1}}%
   {\XXint\scriptstyle\scriptscriptstyle{#1}}%
   {\XXint\scriptscriptstyle\scriptscriptstyle{#1}}%
   \!\int}
\def\XXint#1#2#3{{\setbox0=\hbox{$#1{#2#3}{\int}$}
     \vcenter{\hbox{$#2#3$}}\kern-.5\wd0}}
\def\ddashint{\Xint=}
\def\dashint{\Xint-}

%% TIKZ
 \usepackage{tikz}
%%%<
\usepackage{verbatim}
%\usepackage[active,tightpage]{preview}
\usetikzlibrary{arrows,shapes,positioning}
\usetikzlibrary{decorations.markings}
\usetikzlibrary{calc}
\usetikzlibrary{arrows}

\tikzstyle arrowstyle=[scale=1]
\tikzstyle directed=[postaction={decorate,decoration={markings,
    mark=at position .65 with {\arrow[arrowstyle]{stealth}}}}]
\tikzstyle reverse directed=[postaction={decorate,decoration={markings,
    mark=at position .65 with {\arrowreversed[arrowstyle]{stealth};}}}]


%\documentclass[aps,prl,twocolumn,superscriptaddress,nofootinbib,amssymb]{revtex4-1}
%
%\usepackage{graphicx}
%\usepackage{amsmath}
%\usepackage{times}
%\usepackage{ulem}
%\usepackage{dcolumn}

\bibliographystyle{apsrev4-1}



\renewcommand{\thefootnote}{\fnsymbol{footnote}}
\newcommand{\tio}{TiO$_2$ }
\renewcommand{\figurename}{Figure}
\renewcommand{\thefigure}{S\arabic{figure}}
\renewcommand{\theequation}{S\arabic{equation}}
\renewcommand{\thesection}{S\arabic{section}}
%\pagestyle{empty}

\newcommand{\vect}[1]{\boldsymbol{#1}}
\newcommand{\angstrom}{\textup{\AA}}


\begin{document}

\title{Angle-Resolved Photoemission Spectroscopy of Tetragonal CuO: Evidence for Intralayer Coupling Between Cupratelike Sublattices -- Supplementary Information}

\author{S. Moser}\affiliation{Ecole Polytechnique F\'ed\'erale de Lausanne (EPFL), Institut de Physique des Nanostructures, CH-1015 Lausanne, Switzerland}\affiliation{Advanced Light Source (ALS), Berkeley, California 94720, USA}

%\author{M. Grioni}\affiliation{Ecole Polytechnique F\'ed\'erale de Lausanne (EPFL), Institut de Physique des Nanostructures, CH-1015 Lausanne, Switzerland}



\maketitle

\begin{widetext}
\chapter{Resonant inelastic x-ray scattering}\label{ch2}

\section{Summary}
In the following chapter, we give a brief introduction to resonant inelastic x-ray scattering (RIXS) in the framework of the Kramer-Heisenberg formalism. We show that the spectral weight of RIXS can be reduced to matrix elements mostly sensitive to the on site properties of the wave functions involved. Based on this finding and the Wigner-Eckart theorem, we develop a formalism to estimate the relative spectral weight of transitions to final states with different total symmetries. We further briefly address the problem of self absorption in bulk systems and thin films.

\section{Introduction}

During the past decade, resonant inelastic x-ray scattering (RIXS) became increasingly popular, especially in the domain of correlated electron physics \cite{Ament2011}. Even if more and more spectrometers with higher and higher resolution are under construction all over the planet,\thefootnote{E.g. high resolution spectrometer ``Centurion'' ($R=100000$) and a high throughput spectrometer ``Viking'' ($R=5000$) are currently under construction at NSLS II, \cite{NSLSprojects}.} the available machine time cannot satisfy the increasing demand for experimental shifts. E.g. ``the overbooking factor (i.e., the ratio of requested shifts to available shifts) at each station of the ADRESS beamline is typically $>4$'' \cite{slswebpage}. For a researcher, every minute of granted beam time is obviously precious and the experiment should thus be subject to a careful preparation. In contrast to ARPES, RIXS is a second order process and the underlying cross sections thus very low, which results in typical data acquisition times on the order of hours - with respect to minutes in ARPES - for one spectrum. Performing time consuming $\vect{k}$-space mapping experiments therefore requires an optimized experimental geometry with respect to the sample \textit{and} the desired effect. The experimentalist should be able to perform an ``educated guess'' of what geometry - e.g. what polarization - and what excitation energy to use to observe the desired effect.

Similar to RAMAN spectroscopy, careful consideration of the experimental geometry with respect to the the point group symmetry of the sample therefore is important. So is the detailed knowledge about the final state properties of the related x-ray absorption spectra.

Even if these fundamental considerations sound trivial, the situation becomes less obvious for the ``less experienced'' researcher typically carrying out the experiment. Rather than resuming a complete theory of RIXS, the following chapter aims to depict some useful tools to find a promising geometry.\thefootnote{For a more thorough introduction to RIXS, the reader is referred to refs.~\onlinecite{Schulke} and \onlinecite{Altarelli}.}

\section{The RIXS cross section}

In the Kramers--Heisenberg picture, the RIXS amplitude is described as~\cite{Blume}

\begin{equation}\label{eq: Kramer Heisenberg}
F(h\nu,h\nu')=\sum_f\left|\sum_i\frac{\langle f|\hat{\vect{T}}_2 |i\rangle \langle i|\hat{\vect{T}}_1|g\rangle}{E_i-E_g-h\nu-i\Gamma_i}\right|^2\delta(h\nu-h\nu'+E_g-E_f)~,
\end{equation}

\noindent where $\hat{\vect{T}}_{1/2} = \vect{\epsilon}_{1/2}\cdot \hat{\vect{p}} ~ e^{\pm i\vect{k}_2\vect{r}} $ is the transition operator for absorption (emission) of a photon with polarization $\vect{\epsilon}_1$ ($\vect{\epsilon}_2$), energy $h\nu$ ($h\nu'$) and wavevector $\vect{k}_1$ ($\vect{k}_2$). $| g\rangle$, $| i\rangle$ and $| f \rangle$ are the ground-, the intermediate and the final state in the RIXS process and $E_g$, $E_i$ and $E_f$ their corresponding energies.

In the absorption process, a core electron is excited above the Fermi level, forming an excited core-hole electron state $|i\rangle$. In the subsequent de-excitation process, an electron refills the core-hole, leaving the system either in its groundstate $|f\rangle=|g\rangle$ (resonant elastic scattering) or in some excited state $|f\rangle\neq|g\rangle$ (resonant inelastic scattering).

From Eq.~\ref{eq: Kramer Heisenberg} we immediately can see the main difference between RIXS and non-resonant x-ray scattering $\langle f|H_{int}|g\rangle$: only the atoms absorbing the photon $h\nu$, are directly involved in the resonance process and contribute coherently to the RIXS amplitude. This intimate connection between the excitation and de-excitation process will be the main justification for the symmetry selectivity of both absorption and re-emission process exploited in the following sections.

We now assume that an electron interacts with only one photon at a time and that Coulomb interaction does not significantly perturb the other electrons during the excitation and de-excitation process. This ``frozen core approximation'' is valid only in case the excited charge carriers are sufficiently delocalized and therefore screening effects only marginally affect the other electrons, i.e. if the excitations concern valence states. The many particle wave functions $| g\rangle$ and $| f \rangle$ can thus be approximated as single particle wave functions. In an ordered system, initial and final states are naturally described by Bloch functions of the form

\begin{eqnarray}
| g\rangle & = & e^{i\vect{k}_i \cdot \vect{r}} v_{\vect{k}_i}(\vect{r})\nonumber\\
| f\rangle & = & e^{i\vect{k}_f \cdot \vect{r}} c_{\vect{k}_f}(\vect{r})
\end{eqnarray}

\noindent where $v_{\vect{k}_i}$ and $c_{\vect{k}_i}$ denote lattice periodic functions of the valence and conduction band, respectively. The intermediate state -- strongly localized by the Coulomb force in between core-hole and electron -- may be represented by an atomic wave function with core level index $n$ and centered at atomic position $\vect{R}_j$

\begin{eqnarray}
| i\rangle & = & \Psi_n(\vect{r}-\vect{R}_j)~.
\end{eqnarray}

We now can calculate the matrix elements of Eq.~\ref{eq: Kramer Heisenberg} more explicitly. Making use of the periodicity of $v_{\vect{k}_i}$ and $c_{\vect{k}_i}$ and performing a coordinate transformation $\vect{r}\rightarrow \vect{x}-\vect{R}_j$ and $d^3\vect{x}=d^3\vect{r}$, we obtain

\begin{eqnarray}
\langle f|\hat{\vect{T}}_2 |i\rangle &=& \int e^{-i\vect{k}_f\cdot\vect{r}}c^*_{\vect{k}_f}(\vect{r}) \vect{\epsilon}_2\cdot \hat{\vect{p}}_2 e^{-i \vect{k}_2 \vect{r}}\Psi_n(\vect{r}-\vect{R}_j)d^3\vect{r} \nonumber \\
&=& \int e^{-i\vect{k}_f\cdot(\vect{x}+\vect{R}_j)}c^*_{\vect{k}_f}(\vect{x}) \vect{\epsilon}_2\cdot \hat{\vect{p}}_2 e^{-i \vect{k}_2 (\vect{x}+\vect{R}_j)}\Psi_n(\vect{x})d^3\vect{x}\nonumber\\
&=& e^{-i(\vect{k}_f+\vect{k}_2)\cdot\vect{R}_j}
\int e^{i\vect{k}_f\cdot\vect{x}}c^*_{\vect{k}_f}(\vect{x}) \vect{\epsilon}_2\cdot \hat{\vect{p}}_2 e^{-i \vect{k}_2 \vect{x}}\Psi_n(\vect{x})d^3\vect{x}
\end{eqnarray}

\noindent and in a similar way

\begin{eqnarray}
\langle i|\hat{\vect{T}}_1 |g\rangle &=&  e^{i(\vect{k}_i+\vect{k}_1)\cdot\vect{R}_j}
\int \Psi_n(\vect{x})  \vect{\epsilon}_1\cdot \hat{\vect{p}}_1 e^{i \vect{k}_1 \vect{x}} e^{i\vect{k}_i\cdot\vect{x}}v_{\vect{k}_i}(\vect{x}) d^3\vect{x}
\end{eqnarray}

Carrying out the sum $\sum_i$ partially over all intermediate state lattice sites $\vect{R}_j$ in Eq.~\ref{eq: Kramer Heisenberg} and approximating $\sum_{\vect{R}}$ as an integral $\sum_{\vect{R}}\rightarrow\frac{N}{V}\int d^3\vect{R}$, we can simplify

\begin{equation}
\sum_{\vect{R}_j} e^{i(\vect{k}_i+\vect{k}_1-\vect{k}_f-\vect{k}_2)\cdot\vect{R}_j}=\frac{N}{V}\delta(\vect{k}_i+\vect{k}_1-\vect{k}_f-\vect{k}_2)~,
\end{equation}

\noindent and find the conservation of total momentum. This important finding is especially exploited in RIXS experiments used to determine the dispersion relation of low energy excitations like spinwaves \cite{Guarise2010,Braicovich2009}.

We now write the periodic functions $v_{\vect{k}_i}$ and $c_{\vect{k}_i}$ in terms of maximally localized Wannier functions $\Phi_n(\vect{r}-\vect{R})$

\begin{eqnarray}
v_{\vect{k}_i} & = & e^{-i \vect{k}_i\cdot\vect{r}}\sum_{\vect{R}}e^{i \vect{k}_i\cdot\vect{R}}\Phi_n(\vect{r}-\vect{R})\\
c^*_{\vect{k}_f} & = & e^{-i \vect{k}_f\cdot\vect{r}}\sum_{\vect{R}}e^{i \vect{k}_f\cdot\vect{R}}\Phi_n(\vect{r}-\vect{R})
\end{eqnarray}

\noindent and find the remaining integrals to be of the form

\begin{eqnarray}
\int e^{i\vect{k}_f\cdot\vect{x}}c^*_{\vect{k}_f}(\vect{x}) \vect{\epsilon}_2\cdot \hat{\vect{p}}_2 e^{-i \vect{k}_2 \vect{x}}\Psi_n(\vect{x})d^3\vect{x} & = &\nonumber\\
\int e^{i\vect{k}_f\cdot\vect{x}}e^{-i \vect{k}_f\cdot\vect{x}}\sum_{\vect{R}}e^{i \vect{k}_f\cdot\vect{R}}\Phi_n(\vect{x}-\vect{R}) \vect{\epsilon}_2\cdot \hat{\vect{p}}_2 e^{-i \vect{k}_2 \vect{x}}\Psi_n(\vect{x})d^3\vect{x}\nonumber&=&\\
\sum_{\vect{R}}e^{i \vect{k}_f\cdot\vect{R}} \int \Phi_n(\vect{x}-\vect{R}) \vect{\epsilon}_2\cdot \hat{\vect{p}}_2 e^{-i \vect{k}_2 \vect{x}}\Psi_n(\vect{x})d^3\vect{x}
\end{eqnarray}

\noindent and

\begin{eqnarray}
\int \Psi_n(\vect{x})  \vect{\epsilon}_1\cdot \hat{\vect{p}}_1 e^{i \vect{k}_1 \vect{x}} e^{i\vect{k}_i\cdot\vect{x}}v_{\vect{k}_i}(\vect{x}) d^3\vect{x} &=&\nonumber\\
\sum_{\vect{R}}e^{\vect{k}_i\cdot\vect{R}} \int \Psi_n(\vect{x})  \vect{\epsilon}_1\cdot \hat{\vect{p}}_1 e^{i \vect{k}_1 \vect{x}}
\Phi_n(\vect{x}-\vect{R}) d^3\vect{x}
\end{eqnarray}

Assuming that the intermediate state wave function is strongly localized by the core-hole electron attraction, the major contribution of the sum $\sum_{\vect{R}}$ is covered by the on site term $\vect{R}=\vect{0}$. We further notice that the matrix elements of the RIXS process - as similar to ARPES - can again be written in terms of local integrals. We will thus in the following make use of the symmetry properties of the local Hamiltonian to obtain an ``educated guess'' of the expected RIXS amplitude.


\section{Momentum transfer}

\begin{figure*}[htbp]
\begin{center}
        \includegraphics[width=\columnwidth]{RIXSTheory/RixsSetupGeneralPi.pdf}
\caption{\label{fig: RixsSetupGeneralPi} Typical RIXS experimental configuration as employed at SAXES at SLS or BL07LSU at SPring 8. The sample is represented by its natural coordinate system. The scattering angle $\tau$ is kept fixed while the momentum transfer direction is varied by changing $\alpha$.}
\end{center}
\end{figure*}


The typical experimental geometry (e.g. SAXES at SLS or BL07LSU at SPring 8) is shown in Fig.~\ref{fig: RixsSetupGeneralPi}. The wavevector of the ingoing light $\vect{k}_1$ can be expressed as a function of its energy $h\nu$ and the angles of incidence $\alpha$ and $\phi$ and we find

\begin{eqnarray}
\vect{k}_1&=&-\frac{h\nu}{\hbar c}\left(
                  \begin{array}{ccc}
                    \cos\phi & -\sin\phi & 0 \\
                    \sin\phi &\cos\phi &  0 \\
                     0 &0 & 1 \\
                  \end{array}
                \right)\cdot
                \left(
                  \begin{array}{ccc}
                    1 & 0 & 0 \\
                    0 &\cos(\alpha) &  -\sin(\alpha) \\
                     0 &\sin(\alpha) & \cos(\alpha) \\
                  \end{array}
                \right)\cdot\left(
                         \begin{array}{c}
                           0 \\
                           0 \\
                           1 \\
                         \end{array}
                       \right)\nonumber\\&=&\frac{h\nu}{\hbar c}
                       \left(\begin{array}{c} -\sin(\alpha)\sin\phi\\ \sin(\alpha)\cos\phi\\-\cos(\alpha) \end{array}\right)~.
\end{eqnarray}

The outgoing light direction is related to the incoming light direction $\alpha$ and $\phi$ and the scattering angle $\tau$. By replacing $\alpha$ with $\tau+\alpha-\pi$ in above formula we find

\begin{eqnarray}
\vect{k}_2&=&\frac{h\nu}{\hbar c}
                       \left(\begin{array}{c} -\sin(\tau+\alpha)\sin\phi\\ \sin(\tau+\alpha)\cos\phi\\-\cos(\tau+\alpha) \end{array}\right)~.
\end{eqnarray}

Assuming $h\nu\sim h\nu'$, the momentum transfer to the sample $\vect{q}=\vect{k}_f-\vect{k}_i=\vect{k}_1-\vect{k}_2$ is expressed as

\begin{eqnarray}\label{eq: momentum transfer}
\vect{q}&=&\frac{h\nu}{\hbar c}
                       \left(\begin{array}{c}
                       -\sin\phi(\sin\alpha-\sin(\tau+\alpha))\\ \cos\phi(\sin\alpha-\sin(\tau+\alpha))\\
                       -\cos\alpha+\cos(\tau+\alpha) \end{array}\right)~.
\end{eqnarray}

The in plane momentum $q_{\|}=|\sin\alpha-\sin(\tau+\alpha)|$ is zero for $\alpha= \arccos\left(\frac{1-\cos\tau}{\sqrt{2-2\cos\tau}}\right)$, which is $\alpha=45^{\circ}$ for a scattering geometry $\tau=90^{\circ}$ and $\alpha=25^{\circ}$ for a scattering geometry $\tau=130^{\circ}$, i.e. the specular geometries. Since $\alpha$ is confined by the sample surface, the out of plane component $q_{\perp}=|-\cos\alpha+\cos(\tau+\alpha)|$ is always nonzero.


\section{Basis functions of the tetrahedral system}

The systems studied in this work, anatase TiO$_2$ as well as T-CuO are both systems with tetrahedral point group symmetry. The following concepts are consequently demonstrated for $D_{4h}$. The irreducible representations and standard basis functions $|\Gamma,\gamma\rangle$ of  $D_{4h}$ are tabulated in Tab.~\ref{tab: D4h basis functions} \cite{Koster}.

\begin{table}[htbp]
\begin{center}
\begin{tabular}{c|c|c|c}
  stand. & \backslashbox[3mm]{$\Gamma$}{$\gamma$} & $1$ & $2$ \\
  \hline
   \hline
  $A_{1g}$ & $1$ & $x^2+y^2+z^2$ &  \\
  $A_{2g}$ & $2$ & $xp_y-yp_x$   &  \\
  $B_{1g}$ & $3$ & $x^2-y^2$     &  \\
  $B_{2g}$ & $4$ & $xy$ &  \\
  $E_{g}$  & $5$ & $yp_z-zp_y$ & $zp_x-xp_z$ \\
   \hline
  $A_{1u}$ & $1$ & $(x^2 - y^2)xyz$  &  \\
  $A_{2u}$ & $2$ & $z$ &  \\
  $B_{1u}$ & $3$ & $xyz$ &  \\
  $B_{2u}$ & $4$ & $(x^2 - y^2)z$  &  \\
  $E_{u}$  & $5$ & $x$ & $y$ \\
\end{tabular}
\caption{\label{tab: D4h basis functions} Definition of the irreducible representations and standard basis functions $|\Gamma,\gamma\rangle$ of $D_{4h}$. For example, the component of irreducible representation $E_{u}$ transforming as $y$ is labeled $|5,1\rangle$ in Griffith notation.  }
\end{center}
\end{table}

The basis functions are labeled in Griffith's notation $|\Gamma,\gamma\rangle$ as used later in this work \cite{Griffith}. The usefulness of this notation will become obvious by analogy to a spherical system: its basis functions (e.g. of an atomic Hamiltonian), are given by the spherical harmonics $|l,m\rangle$, where $m=\{-l...l\}$. All spherical harmonics with a fixed $l$ belong to the same irreducible representation of the symmetry group of rotations around a fixed point. Varying $m$ selects distinct components, i.e. the basis function of this representation.

In a non spherical system, $|l,m\rangle$ are no proper basis functions anymore. Their place is now taken by the irreducible representations of the point group in focus, labeled by $\Gamma$. Their components - which can be transformed into one another by the symmetry operations of the group - are labeled by $\gamma$.

In $D_{4h}$ for example, the function $x$ is a component of irreducible representation $E_{u}$. In Griffith's notation this corresponds to $|5,1\rangle^-$ where the ``$-$'' sign indicates the parity. By a simple symmetry operation, e.g. a rotation of $90^{\circ}$ around the $z$-axis, $x$ can be transformed into function $y$, in Griffith's notation $|5,2\rangle^-$, and therefore a component of the same irreducible representation. One can thus use $\Gamma$ and $\gamma$ analogously to the orbital quantum numbers $l$ and $m$ of atomic physics.


\section{Dipole approximation in RIXS}

The wavelength of the soft x-rays we use in this work is large compared to the typical extension of the excited core-hole (e.g. $100$~eV$\sim12.4~\angstrom$). The transition operator $\hat{\vect{T}}$ can thus be approximated as the leading order term of a Taylor series

\begin{equation}
\hat{\vect{T}}=\frac{e}{m c}e^{i\vect{k}\vect{r}}\vect{\epsilon}\cdot \hat{\vect{p}}  \sim \frac{e}{m c} \vect{\epsilon}\cdot \hat{\vect{p}}+...=-i \frac{\hbar e}{m c} \vect{\epsilon}\cdot \hat{\vect{\nabla}}
\end{equation}

The components $T_x\propto\partial/\partial x$, $T_y\propto\partial/\partial y$, $T_z \propto\partial/\partial z$ of this dipole operator $\hat{\vect{T}}$ transform like the functions $x$, $y$ and $z$. In analogy to the work of Matsubara et al. \cite{Matsubara2000,Nakazawa2000,Matsubara2002,Ogasawara2004} and following the group theoretical algebra developed by Tanabe and Sugano \cite{Tanabe}, Tanabe and Kamimura \cite{Kamimura}, Griffith \cite{Griffith} as well as Fano and Racah \cite{Fano}, we will now make use of these symmetry properties to reduce the matrix elements in Eq.~\ref{eq: Kramer Heisenberg}.\thefootnote{Note that the following concepts are neither restricted to the point group symmetry $D_{4h}$, nor to the dipole approximation. In certain situations, it may we useful to estimate the quadropole terms contributing to the RIXS cross section. In this case the components of the quadropole operator have to be considered. In $D_{4h}$, these are transforming as $z^2$, $x^2-y^2$, $xy$ and $xz$/$yz$, and thus represented by irreducible representations $A_{1g}$, $B_{1g}$, $B_{2g}$ and $E_g$.}

%**************************************
%If we have two sets of functions  $f_{\alpha}^a$ and $g_{\beta}^b$, then the sets of products $f_{\alpha}^ag_{\beta}^b$ span the direct product $ab$ of the two constituent representations, which may be reducible. Certain linear combinations $h_{\gamma}^c$ span these irreducible representations:
%
%\begin{equation}
%h_{\gamma}^c=\sum_{\alpha\beta}\langle ab\alpha\beta|abc\gamma\rangle f_{\alpha}^ag_{\beta}^b
%\end{equation}
%
%\begin{eqnarray}
%\sum_{\alpha\beta}\langle abc\gamma|ab\alpha\gamma\rangle\langle ab\alpha\beta|abc'\gamma'\rangle&=&\delta_{cc'}\delta_{\gamma\gamma'}\\
%\sum_{\alpha\beta}\langle ab\alpha\beta|abc\gamma\rangle\langle abc\gamma|ab\alpha'\beta'\rangle&=&\delta_{\alpha\alpha'}\delta_{\beta\beta'}
%\end{eqnarray}
%
%where the first equation is true only if $c$ is contained in the direct product $ab$...
%
%\begin{equation}
%\langle a \alpha|g_{\beta}^b|a'\alpha'\rangle=n^{-1}\langle a||g^b||a'\rangle\langle b a'a\alpha|ba'\beta\alpha'\rangle
%\end{equation}
%
%(lower case english letters name the group representation and roughly corresponding Greek letters name their components, the a$^{th}$ component of a set of functions $f$ transforming according to the irreducible representation $a$ will be written as $f_{\alpha}^a$, or $|a\alpha\rangle$
%


%**************************************


%we can define $\langle a||g^b||a'\rangle=0$ whenever $\delta(a,b,a')=0$ (or in other words if $c$ is in the direct product $ab$)


We start from the Wigner-Eckart theorem, which in Griffith's notation \cite{Griffith} takes the form

\begin{equation}\label{eq: Wigner-Eckart}
\langle \Gamma \gamma | \hat{\vect{O}}_{\overline{\gamma}}^{\overline{\Gamma}} |\Gamma'\gamma' \rangle =\langle \Gamma  || \hat{\vect{O}}^{\overline{\Gamma}} ||\Gamma'\rangle \left(
                                                                             \begin{array}{ccc}
                                                                               \Gamma & \Gamma' & \overline{\Gamma}\\
                                                                                \gamma& \gamma' & \overline{\gamma} \\
                                                                             \end{array}
                                                                           \right)
\end{equation}

\noindent
where the matrix element of the operator $\hat{\vect{O}}$ transforming as component $\overline{\gamma}$ of representation $\overline{\Gamma}$ connects states transforming as components $\gamma$ and $\gamma'$ of the representations $\Gamma$ and $\Gamma'$ respectively. Relation \ref{eq: Wigner-Eckart} holds strictly only for real components. The double-bar reduced matrix element $\langle ... ||...||... \rangle$ is independent of the component labels $\gamma$, $\gamma'$ and $\overline{\gamma}$ and can be conveniently defined as \cite{Griffith}

\begin{equation}
\langle \Gamma  || \hat{\vect{O}}^{\overline{\Gamma}} ||\Gamma'\rangle\equiv c^{\Gamma\Gamma'\overline{\Gamma}}_{\gamma\gamma'\overline{\gamma}}~\delta_{\Gamma, \Gamma' \otimes \overline{\Gamma}}:=\left\{\begin{array}{cc}
c^{\Gamma\Gamma'\overline{\Gamma}}_{\gamma\gamma'\overline{\gamma}} & \forall ~\Gamma \in \{\overline{\Gamma}\otimes \Gamma'\} \\
                                                                         0 & \textnormal{otherwise}
                                                                       \end{array}
                                                                       \right.~,
\end{equation}

\noindent where $c^{\Gamma\Gamma'\overline{\Gamma}}_{\gamma\gamma'\overline{\gamma}}$ are arbitrary coefficients adapted to the explicit problem to solve. In other words, $\langle ... ||...||... \rangle$ can only be nonzero if the total symmetry is conserved, i.e. if the final state has a symmetry which is contained in the direct product of initial state symmetry and the symmetry of the operator $\hat{\vect{O}}$.

The tensor $\bigl( \begin{smallmatrix}
  \Gamma & \Gamma' &\overline{\Gamma}\\
  \gamma & \gamma' &\overline{\gamma}
\end{smallmatrix} \bigr)$ contains coupling coefficients determined by the point group of the system -- analogous to the Wigner $3$-$j$ symbols ($\propto$ Clebsch Gordan coefficients) in spherical symmetry $SO_3$. For the most common point groups, these are tabulated e.g. in Refs.~\onlinecite{Griffith} and \onlinecite{Koster} or online \cite{Snoke}.

As discussed in the previous section, the group representations $\Gamma$, $\Gamma'$ and $\overline{\Gamma}$ ``act'' as the orbital quantum number $l$ and their components $\gamma$, $\gamma'$ and $\overline{\gamma}$ as the magnetic quantum number $m$. Let us illustrate this concept on a simple example: photon absorption in a single atom. The basis function of this problem are the spherical harmonics $|l,m\rangle$, where $m$ runs from $-l$ to $l$. The dipole operator $\hat{\vect{O}}=\hat{\vect{T}}$ is represented by $|l_T,m_T\rangle=|1,0\rangle$ for linear polarized light and $|1,\pm1\rangle$ for circular polarized light. The transition propability to excite the initial state electron $|l_g,m_g\rangle$ into a final state $|l_f,m_f\rangle$, is given by the dipole matrix element $\langle l_f,m_f|T|l_g,m_g\rangle$. The addition of angular momentum imposes that adding angular momentum $l_T=1$ to $l_g$ can only give final state results $l_f=l_g-1$, $l_g$ and $l_g+1$. Further, the total parity $w=(-1)^l$ has to be conserved, i.e. $w_f=(-1)^{l_f}=(-1)^{l_T}(-1)^{l_g}=-w_g$ and therefore $l_f\neq l_g$. These arguments describe the action of the reduced matrix element $\langle ... ||...||... \rangle$. The remaining component $m_g$ depends on the Clebsch Gordon coefficients, which give non-zero contributions only if $m_f=m_g+m_T$. We thus recover the well known selection rules
\begin{eqnarray}
\Delta l &=&\pm1\nonumber\\
\Delta m &=&0,~\pm1~.\\
\end{eqnarray}
In a non-spherical system, the matrix element of Eq.~\ref{eq: Wigner-Eckart} analogously will only be non-zero, if $\Gamma$ is contained in the direct product $\overline{\Gamma}\otimes \Gamma'$ \textit{and} if the coefficients $\bigl( \begin{smallmatrix}
  \Gamma & \Gamma' &\overline{\Gamma}\\
  \gamma & \gamma' &\overline{\gamma}
\end{smallmatrix} \bigr)$ are nonzero. Given that in a general experimental configuration, the dipole operator $\hat{\vect{T}}$ can have components of several different irreducible representations, the total matrix element can be expressed as the weighted sum of all independent partial components of the operator $\hat{\vect{T}}$. The total transition matrix element therefore involves all nonzero projections of the final state onto a linear combination of the form

\begin{eqnarray}
|\Gamma\gamma\rangle&=&\hat{\vect{O}}_{\overline{\gamma}}^{\overline{\Gamma}}|\Gamma'\gamma'\rangle\nonumber\\
&=&\underbrace{\sum_{\Gamma'',\gamma''}|\Gamma''\gamma''\rangle\langle \Gamma'' \gamma''|}_{\equiv1}\hat{\vect{O}}_{\overline{\gamma}}^{\overline{\Gamma}}|\Gamma'\gamma'\rangle\nonumber\\
&=&\sum_{\Gamma'',\gamma''}|\Gamma''\gamma''\rangle \langle \Gamma''  || \hat{\vect{O}}^{\overline{\Gamma}} ||\Gamma'\rangle \left(
                                                                             \begin{array}{ccc}
                                                                               \Gamma'' & \Gamma' & \overline{\Gamma} \nonumber\\
                                                                                \gamma''& \gamma' & \overline{\gamma} \\
                                                                             \end{array}
                                                                           \right)\nonumber\\
&=&\sum_{\Gamma'',\gamma''}|\Gamma''\gamma''\rangle c^{\Gamma''\Gamma'\overline{\Gamma}}_{\gamma''\gamma'\overline{\gamma}}~\delta_{\Gamma'',\overline{\Gamma}\otimes \Gamma'} \left(
                                                                             \begin{array}{ccc}
                                                                               \Gamma'' & \Gamma' & \overline{\Gamma} \\
                                                                                \gamma''& \gamma' & \overline{\gamma} \\
                                                                             \end{array}
                                                                           \right)~,
\end{eqnarray}

\noindent where the completeness of the orthonormal basis set $\{|\Gamma\gamma\rangle\}$ is exploited.

%\begin{eqnarray}
%\sum_{a,\alpha}\langle a \alpha | g_\beta^b |a'\alpha' \rangle&=&\sum_{a,\alpha}
%\delta_{a,b\otimes a'}~c_{\alpha'}V\left(
%                                                                             \begin{array}{ccc}
%                                                                               a & a' & b \\
%                                                                                \alpha& \alpha' & \beta \\
%                                                                             \end{array}
%                                                                           \right)~.
%\end{eqnarray}



\subsection{Symmetry selectivity in the RIXS excitation process}\label{subsec: Symmetry selectivity in the excitation process}


\begin{figure*}[htbp]
\begin{center}
    \parbox[]{0.49\textwidth}{
        \includegraphics[width=0.49\textwidth]{RIXStheory/RixsSetupSigma.pdf}
    }
    \parbox[]{0.49\textwidth}{
        \includegraphics[width=0.49\textwidth]{RIXStheory/RixsSetupPi.pdf}
    }
\caption{\label{fig6} Typical RIXS experimental configuration as employed at SAXES at SLS or BL07LSU at SPring 8. The sample is represented by its natural coordinate system. The scattering angle $\tau$ is kept fixed while the momentum transfer direction is varied by changing $\alpha$. \textbf{(left)} $\sigma$-polarized configuration (polarized, $\xi=0$), \textbf{(right)} $\pi$-polarized configuration (depolarized, $\xi=\pi/2$). Note that the polarization of the outgoing light in general can contain polarized and depolarized components. The azimuthal orientation $\phi$ is typically fixed once the sample is mounted.}
\end{center}
\end{figure*}



Let us now apply the above findings to a RIXS experiment done on a system of $D_{4h}$ point group symmetry, like the perovskite oxides, the cuprates or titanates discussed in this thesis. The typical experimental geometry (e.g. SAXES at SLS or BL07LSU at SPring 8) is again shown in Fig.~\ref{fig6}, for simplicity, we keep the scattering plane fixed in $yz$. The coordinate system is adapted to the basis functions describing the sample, i.e. the $z$ axis is oriented along its surface normal.

\begin{figure*}
\begin{center}
\begin{tikzpicture}

    % define coordinates
    \coordinate (O) at (0,0) ;
    \coordinate (OO) at (0,-2) ;
    \coordinate (A) at (-3,0) ;
    \coordinate (AA) at (-5.25,1.5) ;
    \coordinate (mAA) at (3,-4) ;
    \coordinate (B) at (3,0) ;
    \coordinate (BB) at (4.5,1) ;
    \coordinate (BBB) at (5.25,1.5) ;
    \coordinate (BBBB) at (6,2) ;

    % media
    %\fill[blue!25!,opacity=.3] (-4,0) rectangle (4,2.5);
    \fill[blue!60!,opacity=.3] (-4,0) rectangle (4,-4);
    %\node[] at (-1.5,2) {Vacuum};
    %\node[] at (-1.5,-3) {Solid};

    % axis
    \draw[dash pattern=on5pt off3pt] (-0.5,-4) -- (-0.5,3) ;
        \draw[dash pattern=on5pt off3pt] (0.5,-4) -- (0.5,3) ;
    \draw[dash pattern=on5pt off3pt] (-3,-4) -- (-3,3) ;
   % \draw[dash pattern=on5pt off3pt] (3,-4) -- (3,3) ;
    \draw[thick, yellow] (0.5,-2.75) -- (0.5,-1.25) ;
    \node[right, yellow] at (-0.3,-3) {$\epsilon_x$};


   \draw (0.75,-2.5) arc (-34:30:1);
    \node[] at (0.5,-2)  {$\tau$};
       \node[] at (3.9,1.5)  {$\tau+\alpha-\pi$};

    % rays
    \draw[thin, dash pattern=on5pt off3pt] (AA) -- (mAA);
    \draw[red,ultra thick,directed] (AA) -- (OO);
      \node[red,right] at (-4.5,1.25) {$h\nu$};

      \draw[yellow,ultra thick,directed] (OO) -- (0.5,-1.25);
            \draw[yellow,ultra thick,directed] (OO) -- (-0.5,-2.75);
   %   \node[red,right] at (-1.5,-0.75) {$l$};




   % \draw[blue,directed,ultra thick] (OO) -- (BBB);
% \node[blue,left] at (1.5,-0.75) {$l'$};

    % angles
    \draw (-3,1) arc (90:145.2:1);
  %  \node[] at (3.9,1.5)  {$\pi-\tau-\alpha$};

 %   \draw (3,2.5) arc (90:33:2.5) ;
    \node[] at (-3.3,0.5)  {$\alpha$};

   \draw (0.75,-2.5) arc (-34:30:1);
    \node[] at (0.5,-2)  {$\tau$};


        \draw[blue,directed,thin, dash pattern=on5pt off3pt] (OO) -- (BBB);
%     \draw[blue,thin] (-2.25,-3.5) -- (BBB);

\node[blue,left] at (6,1.25) {$h\nu'$};
 \draw[dash pattern=on5pt off3pt] (3,-4) -- (3,3) ;


    \draw[dash pattern=on5pt off3pt] (-4,-2.75) -- (4,-2.75) ;
        \draw[dash pattern=on5pt off3pt] (-4,-1.25) -- (4,-1.25) ;
   \draw (3,2.5) arc (90:33:2.5) ;

    \draw[thick,yellow] (-0.5,-2.75) -- (0.5,-2.75) ;
    \node[right,yellow] at (0.5,-2.3) {$\epsilon_z$};

\end{tikzpicture}
\caption{\label{fig: ingoing projection} Projections of a $\pi$-polarization vector onto its principal components $\epsilon_x$ and $\epsilon_z$.}
\end{center}
\end{figure*}

The Kramer Heisenberg formula of Eq.~\ref{eq: Kramer Heisenberg} involves two dipole matrix elements, one for the photon absorption and one for the photon emission process. For the absorption, we are primarily interested in the relative orientation of the incident polarization with respect to the symmetry planes of the sample as sketched in Fig.~\ref{fig: ingoing projection}. Assuming linear polarization, the vector potential of the incoming light can be written as

\begin{eqnarray}\label{eq: polarization vector}
\vect{\epsilon}_1=\left(
          \begin{array}{c}
            \epsilon_x \\
            \epsilon_y \\
            \epsilon_z \\
          \end{array}
        \right)&=&\epsilon\left(
                  \begin{array}{ccc}
                    \cos\phi & -\sin\phi & 0 \\
                    \sin\phi &\cos\phi &  0 \\
                     0 &0 & 1 \\
                  \end{array}
                \right)\cdot
                \left(
                  \begin{array}{ccc}
                    1 & 0 & 0 \\
                    0 &\cos\alpha &  -\sin\alpha \\
                     0 &\sin\alpha & \cos\alpha \\
                  \end{array}
                \right)\cdot\left(
                         \begin{array}{c}
                           \cos\xi \\
                           \sin\xi \\
                           0 \\
                         \end{array}
                       \right)\nonumber\\
                       &=&
                       \epsilon\left(\begin{array}{c} \cos\phi\cos\xi-\cos\alpha\sin\phi\sin\xi\\ \sin\phi\cos\xi+\cos\alpha\cos\phi\sin\xi\\\sin\alpha\sin\xi \end{array}\right)~,
\end{eqnarray}

\noindent where $\alpha$ is the angle in between the surface normal $z$ and the incoming light, and $\xi$ is the polarization angle. $\phi$ is the azimuthal rotation angle of the sample, which will be typically fixed once the sample is mounted. Depending on $(\phi,\alpha,\xi)$, the dipole operator can have projections onto $x$, $y$ and $z$ components and according to Tab.~\ref{tab: D4h basis functions} is contained both in the $E_u(x,y)$ and $A_{2u}(z)$ irreducible representations of $D_{4h}$.

To simplify the discussion, we now keep the scattering plane oriented along a high symmetry plane $yz$ with $\phi=0$, but give the general results in the associated tables. We identify two important experimental configurations:
in the $\sigma$-polarized configuration ($\xi=0$), the polarization vector $\vect{\epsilon}$ is along $x$ and perpendicular to the scattering plane $yz$. The dipole operator $\hat{\vect{T}}_1^\sigma$ representing the incident light thus is contained solely in the $E_u(x)$ ($\rightarrow|51\rangle^-$) representation of $D_{4h}$. In the $\pi$-polarized configuration ($\xi=\pi/2$), the polarization vector lies in the scattering plane $yz$. Now, a fraction $\cos \alpha$ of the photons represented by $\hat{\vect{T}}_1^\pi$ involve transitions of $E_u(y)$ ($\rightarrow|52\rangle^-$), whereas a fraction $\sin \alpha$ involve transitions of $A_{2u}(z)$ ($\rightarrow|21\rangle^-$) symmetry.



\begin{table*}[htbp]
\resizebox{\columnwidth}{!}{
\begin{tabular}{cc|c||cc|c||ccccccc}

  \multicolumn{3}{c||}{$|g\rangle$}     &  \multicolumn{3}{c||}{$T_1$}  &     \multicolumn{6}{c}{$\Gamma''\gamma''$}                                    \\
\hline
           &  &    &&      &&   $A_1$ & $A_2(z)$ &     $B_1(x^2-y^2)$ &      $B_2(xy)$ &   $E(x)$ &   $E(y)$         \\
           \multicolumn{2}{c|}{$\Gamma'\gamma'$}      & $\times$& \multicolumn{2}{c|}{$\overline{\Gamma}\overline{\gamma}$}&$\times$ &  $|11\rangle$ & $|21\rangle$ &     $|31\rangle$ &      $|41\rangle$ &   $|51\rangle$ &   $|52\rangle$    \\

\hline
\hline
   & & & $E(x)$ &  $|51\rangle$& $\sqrt{(\cos\phi\cos\xi-\cos\alpha\sin\phi\sin\xi)^2}$&       0 &   0&       0 &          0 &          1 &          0 &                               \\

   $A_1(x)$& $|11\rangle$  & 1     &   $E(y)$ & $|52\rangle$&$\sqrt{\sin\phi\cos\xi+\cos\alpha\cos\phi\sin\xi)^2}$ &        0 &   0&      0 &          0 &          0 &          1 &            \\

      &     &  & $A_2(z)$ &    $|21\rangle$& $\sqrt{(\sin\alpha\sin\xi)^2}$  &   0 &      1&    0 &          0 &          0 &          0&                           \\

\hline
\end{tabular}}
\caption{\label{tab: Absorption A1g D4h} Symmetry selectivity in the RIXS excitation process from an $A_{1g}$ groundstate in $D_{4h}$ symmetry. The use of this table is explained in detail in the caption of Tab.~\ref{tab: Emission A1g D4h}.}
\end{table*}


\subsubsection{Absorption from an $A_{1g}$ groundstate}

In undoped anatase TiO$_2$, the $d^0$ groundstate of the system $|g\rangle$ is of spherical $A_{1g}$ symmetry and thus can be labeled by basis function $|11\rangle^+$ in Griffith notation. Now we apply the Wigner-Eckart theorem to the absorption process in $\sigma$-polarization in order to calculate the linear combination $|i^{\sigma}\rangle$ associated to all symmetry allowed XAS final state projections $|\Gamma,\gamma\rangle$ with non zero matrix elements.

%\begin{equation}
%|a,\alpha\rangle=\sum_{a',\alpha'}  V\left(
%                                                                             \begin{array}{ccc}
%                                                                               a' & 1 & 5 \\
%                                                                                \alpha'& 1 & 1 \\
%                                                                             \end{array}
%                                                                           \right)  \hat{g}^b |a'\alpha'\rangle=|51\rangle\equiv E_u(x)\nonumber
%\end{equation}

\begin{eqnarray}
|i^{\sigma}\rangle&=&\sum_{\Gamma'',\gamma''} c^{\Gamma''15}_{\gamma''11}~\delta_{\Gamma'',5} \left(
                                                                             \begin{array}{ccc}
                                                                               \Gamma'' & 1 & 5 \\
                                                                                \gamma''& 1 & 1 \\
                                                                             \end{array}
                                                                           \right)|\Gamma''\gamma''\rangle \nonumber\\
&=&c^{515}_{111}|51\rangle^- \nonumber\\
&=&|51\rangle^-
\end{eqnarray}



\noindent  The coefficient $c^{515}_{111}$ is determined by the orientation of the dipole operator $\vect{\epsilon}_1$ with respect to the crystal coordinates, or in other words by the fraction of photons polarized along $x$, namely $c^{515}_{111}=\sqrt{(\cos\phi\cos\xi-\cos\alpha\sin\phi\sin\xi)^2}=1$ for $\phi=\xi=0$.


In an analogous way, we yield

\begin{eqnarray}
|i^{\pi}\rangle&=&\sum_{\Gamma'',\gamma''} \left(c^{\Gamma''15}_{\gamma''12}~\delta_{\Gamma'',5} \left(
                                                                             \begin{array}{ccc}
                                                                               \Gamma'' & 1 & 5 \\
                                                                                \gamma''& 1 & 2 \\
                                                                             \end{array}
                                                                           \right)+
c^{\Gamma''12}_{\gamma''11}~\delta_{\Gamma'',2} \left(
                                                                             \begin{array}{ccc}
                                                                               \Gamma'' & 1 & 2 \\
                                                                                \gamma''& 1 & 1 \\
                                                                             \end{array}
                                                                           \right)\right)|\Gamma''\gamma''\rangle \nonumber\\                                                                      &=& c^{515}_{212} |52\rangle^- +c^{212}_{111} |21\rangle^-  \nonumber\\
&=&  \cos\alpha  |52\rangle^-  + \sin\alpha |21\rangle^-
\end{eqnarray}

\noindent in $\pi$-polarization $\xi=\pi/2$.
Whereas we only find spectral weight from $E_u$ total symmetry final states in $\sigma$-polarization, we can find additional final states of $A_{2u}$ symmetry in $\pi$-polarization. The general result of an arbitrary experimental configuration is presented in Tab.~\ref{tab: Absorption A1g D4h}.


\subsubsection{Absorption from a $B_{2g}$ groundstate}

\begin{table*}[htbp]
\resizebox{\columnwidth}{!}{
\begin{tabular}{cc|c||cc|c||ccccccc}

  \multicolumn{3}{c||}{$|g\rangle$}     &  \multicolumn{3}{c||}{$T_1$}  &     \multicolumn{6}{c}{$\Gamma''\gamma''$}                                    \\
\hline
           &  &    &&      &&   $A_1$ & $A_2(z)$ &     $B_1(x^2-y^2)$ &      $B_2(xy)$ &   $E(x)$ &   $E(y)$         \\
           \multicolumn{2}{c|}{$\Gamma'\gamma'$}      & $\times$& \multicolumn{2}{c|}{$\overline{\Gamma}\overline{\gamma}$}&$\times$ &  $|11\rangle$ & $|21\rangle$ &     $|31\rangle$ &      $|41\rangle$ &   $|51\rangle$ &   $|52\rangle$    \\

\hline
\hline
   & & & $E(x)$ &  $|51\rangle$& $\sqrt{(\cos\phi\cos\xi-\cos\alpha\sin\phi\sin\xi)^2}$&       0 &   0&       0 &          0 &          0 &          1&                               \\

   $B_2(xy)$& $|41\rangle$  & 1     &   $E(y)$ & $|52\rangle$&$\sqrt{\sin\phi\cos\xi+\cos\alpha\cos\phi\sin\xi)^2}$ &        0 &   0&      0 &          0 &          1 &          0 &            \\

      &     &  & $A_2(z)$ &    $|21\rangle$& $\sqrt{(\sin\alpha\sin\xi)^2}$  &   0 &      0&    0 &          0 &          0 &          0&                           \\

\hline
\end{tabular}}
\caption{\label{tab: Absorption B2g D4h} Symmetry selectivity in the excitation process from a ground state $B_{2g}$ in $D_{4h}$ symmetry. The use of this table is explained in detail in the caption of Tab.~\ref{tab: Emission A1g D4h}.}
\end{table*}

Doping anatase TiO$_2$ with oxygen vacancies introduces Ti$^{3+}$ sites with $d_{xy}^1$ character. Locally, the system now can be described by a $3d_{xy}$ ground state with ground state symmetry $B_{2g}$ ($|41\rangle^+)$.

Again, we apply the Wigner-Eckart theorem to the absorption process in $\sigma$- and $\pi$-polarization in order to calculate all symmetry allowed XAS final states:

\begin{eqnarray}
|i^{\sigma}\rangle&=&\sum_{\Gamma'',\gamma''} c^{\Gamma''45}_{\gamma''11}~\delta_{\Gamma'',5} \left(
                                                                             \begin{array}{ccc}
                                                                               \Gamma'' & 4 & 5 \\
                                                                                \gamma''& 1 & 1 \\
                                                                             \end{array}
                                                                           \right)|\Gamma''\gamma''\rangle \nonumber\\
&=&c^{545}_{211}|52\rangle^-
\end{eqnarray}

\noindent where the weight $c^{545}_{211}$ is determined by the fraction of photons causing the transition, namely $c^{545}_{211}=1$.

In an analogous way, we have

\begin{eqnarray}
|i^{\pi}\rangle&=&\sum_{\Gamma'',\gamma''} \left(c^{\Gamma''45}_{\gamma''12}~\delta_{\Gamma'',5} \left(
                                                                             \begin{array}{ccc}
                                                                               \Gamma'' & 4 & 5 \\
                                                                                \gamma''& 1 & 2 \\
                                                                             \end{array}
                                                                           \right)+
c^{\Gamma''42}_{\gamma''11}~\delta_{\Gamma'',2} \underbrace{\left(
                                                                             \begin{array}{ccc}
                                                                               \Gamma'' & 4 & 2 \\
                                                                                \gamma''& 1 & 1 \\
                                                                             \end{array}
                                                                           \right)}_{=0}\right)|\Gamma''\gamma''\rangle \nonumber\\                                                                      &=& c^{545}_{112} |51\rangle^-=\cos\alpha  |51\rangle^-~.
\end{eqnarray}


\subsubsection{Absorption from a $B_{1g}$ groundstate}

\begin{table*}[htbp]
\resizebox{\columnwidth}{!}{
\begin{tabular}{cc|c||cc|c||ccccccc}

  \multicolumn{3}{c||}{$|g\rangle$}     &  \multicolumn{3}{c||}{$T_1$}  &     \multicolumn{6}{c}{$\Gamma''\gamma''$}                                    \\
\hline
           &  &    &&      &&   $A_1$ & $A_2(z)$ &     $B_1(x^2-y^2)$ &      $B_2(xy)$ &   $E(x)$ &   $E(y)$         \\
           \multicolumn{2}{c|}{$\Gamma'\gamma'$}      & $\times$& \multicolumn{2}{c|}{$\overline{\Gamma}\overline{\gamma}$}&$\times$ &  $|11\rangle$ & $|21\rangle$ &     $|31\rangle$ &      $|41\rangle$ &   $|51\rangle$ &   $|52\rangle$    \\

\hline
\hline
   & & & $E(x)$ &  $|51\rangle$& $\sqrt{(\cos\phi\cos\xi-\cos\alpha\sin\phi\sin\xi)^2}$&       0 &   0&       0 &          0 &          0 &          1&                               \\

   $B_1(x^2-y^2)$& $|31\rangle$  & 1     &   $E(y)$ & $|52\rangle$&$\sqrt{\sin\phi\cos\xi+\cos\alpha\cos\phi\sin\xi)^2}$ &        0 &   0&      0 &          0 &          1 &          0 &            \\

      &     &  & $A_2(z)$ &    $|21\rangle$& $\sqrt{(\sin\alpha\sin\xi)^2}$  &   0 &      0&    0 &          0 &          0 &          0&                           \\

\hline
\end{tabular}}
\caption{\label{tab: Absorption B1g D4h} Symmetry selectivity in the excitation process from a $B_{1g}$ ground state in $D_{4h}$ symmetry. The use of this table is explained in detail in the caption of Tab.~\ref{tab: Emission A1g D4h}.}
\end{table*}

In tetragonal CuO, the hole of the $d^9$ groundstate mostly occupies the $d_{x^2-y^2}$ orbital. Tab.~\ref{tab: Absorption B1g D4h} therefore summarizes the same consideration starting from a $B_{1g}$ groundstate. We immediately see that the result is equivalent to the result obtained for $B_{2g}$.

It is important to note that a high exchange coupling $J$ in the correlated cuprates leads to antiferromagnetic ordering. The ground state wave function therefore necessarily needs to take into account the spin and the ground state symmetry is more appropriately described by the according magnetic (Shubnikov) point group.

\subsection{Symmetry selectivity in the re-emission process}

\begin{figure*}
\begin{center}
\begin{tikzpicture}

    % define coordinates
    \coordinate (O) at (0,0) ;
    \coordinate (OO) at (0,-2) ;
    \coordinate (A) at (-3,0) ;
    \coordinate (AA) at (-5.25,1.5) ;
    \coordinate (mAA) at (3,-4) ;
    \coordinate (B) at (3,0) ;
    \coordinate (BB) at (4.5,1) ;
    \coordinate (BBB) at (5.25,1.5) ;
    \coordinate (BBBB) at (6,2) ;

    \fill[blue!60!,opacity=.3] (-4,0) rectangle (4,-4);
    \draw[dash pattern=on5pt off3pt] (3,-4) -- (3,3) ;
    \draw[ultra thick] (0,-2.75) -- (0,-1.25) ;
    \node[right] at (-0.7,-2) {$T_z$};
     \draw[thin,dash pattern=on5pt off3pt] (AA) -- (mAA);
    \draw[red,thin,directed, dash pattern=on5pt off3pt] (AA) -- (OO);
      \node[red,right] at (-4.5,1.25) {$h\nu$};
    \draw[blue,directed,ultra thick] (OO) -- (BBB);
    \draw[blue, dash pattern=on5pt off3pt] (0,-2.75) -- (3.0,-0.75) ;
    \node[blue,left] at (6,1.25) {$h\nu'$};

    \draw[yellow, dash pattern=on5pt off3pt] (-1,0.25) -- (1.5,-3.5) ;
    \draw[yellow, ultra thick] (0,-1.25) -- (0.7,-2.3) ;

    \node[yellow,right] at (0,-1.3) {$|P_{2z}|$};

     \draw[dash pattern=on5pt off3pt] (-3,-4) -- (-3,3) ;

           \draw (-3,1) arc (90:145.2:1);
    \node[] at (-3.3,0.5)  {$\alpha$};

    \draw (0.75,-2.5) arc (-34:30:1);
    \node[] at (0.5,-2)  {$\tau$};
       \node[] at (3.9,1.5)  {$\tau+\alpha-\pi$};

   \draw (3,2.5) arc (90:33:2.5) ;

\end{tikzpicture}
\caption{\label{fig: outgoing projection} Projection $|P_{2z}|$ of the principal component $T_z$ onto the polarization plane of the emitted light.}
\end{center}
\end{figure*}

In the de-excitation process, the system will create photons represented by the dipole operator $\hat{\vect{T}}_2$. The principal components of $\hat{\vect{T}}_2$ are $E_u(x)$, $E_u(y)$ and $A_{2u}(z)$ as exemplarily demonstrated in Fig.~\ref{fig: outgoing projection}. However, only photons with principal components that have a non-zero projections perpendicular to the outgoing light direction will be measured. In order to calculate the number of photons escaping along the outgoing light direction $\vect{k}_2$, we therefore need to calculate the projections of $\vect{e}_x=\{1,0,0\}$, $\vect{e}_y=\{0,1,0\}$ and $\vect{e}_z=\{0,0,1\}$ onto the polarization plane. The outgoing light direction is

\begin{eqnarray}
\frac{\hbar c}{h\nu}\vect{k}_2&=&
                       \left(\begin{array}{c} -\sin(\tau+\alpha)\sin\phi\\ \sin(\tau+\alpha)\cos\phi\\-\cos(\tau+\alpha) \end{array}\right)~,
\end{eqnarray}

The projection of an arbitrary vector $\vect{v}$ onto the polarization plane perpendicular to $\vect{k}_2$ is given by

\begin{equation}
\vect{P}_2=\vect{v}-(\vect{v}\cdot\frac{\vect{k}_2}{k_2})\frac{\vect{k}_2}{k_2}~.
\end{equation}

The absolute projections $|P_{2x}|$, $|P_{2y}|$, $|P_{2z}|$ of $\vect{e}_x=\{1,0,0\}$, $\vect{e}_y=\{0,1,0\}$ and $\vect{e}_z=\{0,0,1\}$ onto the outgoing polarization plane can therefore be easily calculated as

\begin{equation}
\left(\begin{array}{c}
         |P_{2x}|\\
         |P_{2y}|\\
         |P_{2z}|
       \end{array}
\right)=
\left(\begin{array}{c}
         \frac{1}{2}\sqrt{3+\cos(2\phi)+2\cos\left[2(\alpha+\tau)\right]\sin^2\phi}\\
         \frac{1}{2}\sqrt{3-\cos(2\phi)+2\cos\left[2(\alpha+\tau)\right]\cos^2\phi}\\
         \sqrt{\sin^2(\alpha+\tau)}
       \end{array}
\right)~.
\end{equation}





\begin{table*}[htbp]
\resizebox{\columnwidth}{!}{
\begin{tabular}{cc|c||cc|c||ccccccc}

  \multicolumn{3}{c||}{$|i\rangle$}     &  \multicolumn{3}{c||}{$T_2$}  &     \multicolumn{6}{c}{$\Gamma''\gamma''$}                                    \\
\hline
           &  &    &&      &&   $A_1$ & $A_2(z)$ &     $B_1(x^2-y^2)$ &      $B_2(xy)$ &   $E(x)$ &   $E(y)$         \\
           \multicolumn{2}{c|}{$\Gamma'\gamma'$}      & $\times$& \multicolumn{2}{c|}{$\overline{\Gamma}\overline{\gamma}$}&$\times$ &  $|11\rangle$ & $|21\rangle$ &     $|31\rangle$ &      $|41\rangle$ &   $|51\rangle$ &   $|52\rangle$    \\

\hline
\hline
   & & & $E(x)$ &  $|51\rangle$& $\frac{1}{2}\sqrt{3+\cos(2\phi)+2\cos\left[2(\alpha+\tau)\right]\sin^2\phi}$&       $1/\sqrt{2}$ &   0&       $1/\sqrt{2}$ &          0 &          0 &          0 &                               \\

   $E(x)$& $|51\rangle$  & $\sqrt{(\cos\phi\cos\xi-\cos\alpha\sin\phi\sin\xi)^2}$     &   $E(y)$ & $|52\rangle$&$\frac{1}{2}\sqrt{3-\cos(2\phi)+2\cos\left[2(\alpha+\tau)\right]\cos^2\phi}$ &        0 &   $1/\sqrt{2}$&      0 &          $1/\sqrt{2}$ &          0 &          0 &            \\

      &     &  & $A_2(z)$ &    $|21\rangle$& $\sqrt{\sin^2(\alpha+\tau)}$  &   0 &      0&    0 &          0 &          0 &          0&                           \\


\hline
   & & & $E(x)$ &  $|51\rangle$&  $\frac{1}{2}\sqrt{3+\cos(2\phi)+2\cos\left[2(\alpha+\tau)\right]\sin^2\phi} $ &    0 &   $-1/\sqrt{2}$&       0 &          $1/\sqrt{2}$ &          0 &          0 &                     \\

   $E(y)$& $|52\rangle$   &  $\sqrt{(\sin\phi\cos\xi+\cos\alpha\cos\phi\sin\xi)^2}$   &   $E(y)$ & $|52\rangle$&  $\frac{1}{2}\sqrt{3-\cos(2\phi)+2\cos\left[2(\alpha+\tau)\right]\cos^2\phi}$  &     $1/\sqrt{2}$ &   0&       $-1/\sqrt{2}$ &          0 &          0 &          0 &        \\

      &   &  &   $A_2(z)$ &    $|21\rangle$&  $\sqrt{\sin^2(\alpha+\tau)}$ &   0 &      0&    0 &          0 &      0 &          0 &           \\

\hline
  & & &  $E(x)$ &   $|51\rangle$&  $\frac{1}{2}\sqrt{3+\cos(2\phi)+2\cos\left[2(\alpha+\tau)\right]\sin^2\phi}$ &    0 &     0&     0 &          0 &          0 &          $-1$ &          \\

  $A_2(z)$& $|21\rangle$   &$\sqrt{(\sin\alpha\sin\xi)^2}$&        $E(y)$ &    $|52\rangle$& $\frac{1}{2}\sqrt{3-\cos(2\phi)+2\cos\left[2(\alpha+\tau)\right]\cos^2\phi}$   &  0 &    0&      0 &          0 &          $1$ &          0 &           \\

        &  &&   $A_2(z)$ &    $|21\rangle$& $\sqrt{\sin^2(\alpha+\tau)}$ &    0 &      0&    0 &          0 &          0 &          0 &          \\


\hline
\end{tabular}}
\caption{\label{tab: Emission A1g D4h} Symmetry selectivity in the re-emission process after absorption from a spherical ground state $A_{1g}$ in $D_{4h}$ symmetry. Note that parity is omitted. The table consists of three major parts. The left part, here marked by $|i\rangle$, marks the initial state of the transition. The column indicated by ``$\times$'' gives the relative weight of initial state $\Gamma'\gamma'$. The middle part, here indicated by $T_2$, gives the relevant transitions $\overline{\Gamma}\overline{\gamma}$ and their relative weight indicated by ``$\times$''. The right part summarizes all coupling coefficients between initial state and transition operator. To find the spectral weight of a certain final state $|\Gamma'',\gamma''\rangle$, one has to sum over all non zero components of the associated column multiplied by the weights ``$\times$'' of the same row. For final state $|5,2\rangle$ this would be e.g. $\left(-\frac{1}{2}\sqrt{3+\cos(2\phi)+2\cos\left[2(\alpha+\tau)\right]\sin^2\phi}\sqrt{(\sin\alpha\sin\xi)^2}\right)^2$.}
\end{table*}


\subsubsection{Re-emission after absorption from an $A_{1g}$ groundstate}


The dipole allowed intermediate states of the RIXS process calculated in \ref{subsec: Symmetry selectivity in the excitation process} will now be used to apply the Wigner-Eckart theorem a second time, corresponding to the emission process in the RIXS process. The calculation is completely analogous - except that coefficients $c^{\Gamma\Gamma'\overline{\Gamma}}_{\gamma\gamma'\overline{\gamma}}$ are now determined by $\{|P_x|,|P_y|,|P_z|\}$ - and itemized in detail in Tab.~\ref{tab: Emission A1g D4h}.

We find that dipole allowed contributions to the final state in $\sigma$-polarized ($\xi=0$) are described by the linear combination

\begin{eqnarray}
|f^\sigma\rangle&=&\frac{1}{\sqrt{2}}\left[ |11\rangle^++\cos(\tau+\alpha) |21\rangle^+ + |31\rangle^+ +\cos(\tau+\alpha) |41\rangle^+\right]\nonumber
\end{eqnarray}

\noindent and for $\pi$-polarization ($\xi=\pi/2$)

\begin{eqnarray}
|f^\pi\rangle&=&\frac{1}{\sqrt{2}}\left[ \cos\alpha\cos(\tau+\alpha) |11\rangle^+- \cos\alpha |21\rangle^+ - \cos\alpha\cos(\tau+\alpha)|31\rangle^+ + \cos\alpha |41\rangle^+\right]\nonumber\\
&&+\sin\alpha\cos(\tau+\alpha)|51\rangle^+- \sin\alpha |52\rangle^+\nonumber~.
\end{eqnarray}

Note that we again kept $\phi=0$ for ease of discussion. The spectral weight of an isolated final state $|\Gamma,\gamma\rangle$ is finally given by $I\propto |\langle \Gamma,\gamma|f\rangle|^2$ and summarized in Tab.~\ref{tab: mode spectral weight}.

\begin{table*}[htbp]
\resizebox{\columnwidth}{!}{
\begin{tabular}{c|c|c|c|c|c|c|c}
   $\phi$&pol& $A_{1g}$ & A$_{2g}$ & $B_{1g}$ & $B_{2g}$ & $E_g(yz)$ & $E_g(xz)$ \\
    \hline
  0&$\sigma$ & $\frac{1}{2}$ & $\frac{1}{2}\cos^2(\alpha+\tau)$ & $\frac{1}{2}$ & $\frac{1}{2}\cos^2(\alpha+\tau)$ & $0$ & $0$ \\
  0&$\pi$    & $\frac{1}{2}\cos^2\alpha\cos^2(\alpha+\tau)$ & $\frac{1}{2}\cos^2\alpha$ & $\frac{1}{2}\cos^2\alpha\cos^2(\alpha+\tau)$ & $\frac{1}{2}\cos^2\alpha$ & $\cos^2(\alpha+\tau)\sin^2\alpha$ & $\sin^2\alpha$ \\
  \hline
  $\pi/4$&$\sigma$&$\frac{1}{4}(3+\cos\left[2(\alpha+\tau)\right])$ & 0 & 0 & $\frac{1}{4}(3+\cos\left[2(\alpha+\tau)\right])$ &0&0\\
  $\pi/4$&$\pi$& $\frac{1}{4}\cos^2\alpha(3+\cos\left[2(\alpha+\tau)\right])$ & 0 & 0 &$\frac{1}{4}\cos^2\alpha(3+\cos\left[2(\alpha+\tau)\right])$ &$\frac{1}{4}\sin^2\alpha(3+\cos\left[2(\alpha+\tau)\right])$&$\frac{1}{4}\sin^2\alpha(3+\cos\left[2(\alpha+\tau)\right])$\\
\end{tabular}}
\caption{\label{tab: mode spectral weight} Spectral weight of the final states as a function of scattering angle $\tau$ and the incoming light direction $\alpha$ starting from a $A_{1g}$ groundstate.}
\end{table*}

\begin{figure*}[htbp]
\begin{center}
\includegraphics[width=\columnwidth]{RIXStheory/RIXSmodeSelection.pdf}
\caption{\label{fig5} Angular dependence of all dipole allowed final states from a $A_{1g}$ ground state in $\sigma$- and $\pi$-polarized configuration. The results are shown \textbf{(left)} for scattering angles $\tau=90^{\circ}$ and \textbf{(right)} $130^{\circ}$ as used throughout the scope of this thesis. }
\end{center}
\end{figure*}


The expected dependence of all dipole allowed final states on the incidence angle $\alpha$ for $\sigma$ and $\pi$-polarization is plotted in Fig.~\ref{fig5}. For $\phi=0$, final states of $A_{1g}$ and $B_{1g}$ symmetry, behave in the same way. So behave final states of $A_{2g}$ and $B_{2g}$ symmetry. For $\phi=45^{\circ}$, $A_{1g}$ groups with $B_{2g}$ and $A_{2g}$ with $B_{1g}$. Apart from these ``degeneracies'', the behavior of the branches is quite complex, and it is often possible suppress certain final state symmetries with respect to others.


\subsubsection{Re-emission after absorption from a $B_{1g}$ or $B_{2g}$ groundstate}


\begin{table*}[htbp]
\resizebox{\columnwidth}{!}{
\begin{tabular}{cc|c||cc|c||ccccccc}

  \multicolumn{3}{c||}{$|i\rangle$}     &  \multicolumn{3}{c||}{$T_2$}  &     \multicolumn{6}{c}{$\Gamma''\gamma''$}                                    \\
\hline
           &  &    &&      &&   $A_1$ & $A_2(z)$ &     $B_1(x^2-y^2)$ &      $B_2(xy)$ &   $E(x)$ &   $E(y)$         \\
           \multicolumn{2}{c|}{$\Gamma'\gamma'$}      & $\times$& \multicolumn{2}{c|}{$\overline{\Gamma}\overline{\gamma}$}&$\times$ &  $|11\rangle$ & $|21\rangle$ &     $|31\rangle$ &      $|41\rangle$ &   $|51\rangle$ &   $|52\rangle$    \\

\hline
\hline
   & & & $E(x)$ &  $|51\rangle$& $\frac{1}{2}\sqrt{3+\cos(2\phi)+2\cos\left[2(\alpha+\tau)\right]\sin^2\phi}$&       $1/\sqrt{2}$ &   0&       $1/\sqrt{2}$ &          0 &          0 &          0 &                               \\

   $E(x)$& $|51\rangle$  & $\sqrt{\sin\phi\cos\xi+\cos\alpha\cos\phi\sin\xi)^2}$    &   $E(y)$ & $|52\rangle$&$\sqrt{\sin\phi\cos\xi+\cos\alpha\cos\phi\sin\xi)^2}$ &        0 &   $1/\sqrt{2}$&      0 &          $1/\sqrt{2}$ &          0 &          0 &            \\

      &     &  & $A_2(z)$ &    $|21\rangle$& $\sqrt{\sin^2(\alpha+\tau)}$  &   0 &      0&    0 &          0 &          0 &          0&                           \\


\hline
   & & & $E(x)$ &  $|51\rangle$&  $\frac{1}{2}\sqrt{3+\cos(2\phi)+2\cos\left[2(\alpha+\tau)\right]\sin^2\phi} $ &    0 &   $-1/\sqrt{2}$&       0 &          $1/\sqrt{2}$ &          0 &          0 &                     \\

   $E(y)$& $|52\rangle$   &  $\sqrt{(\cos\phi\cos\xi-\cos\alpha\sin\phi\sin\xi)^2}$    &   $E(y)$ & $|52\rangle$&  $\sqrt{\sin\phi\cos\xi+\cos\alpha\cos\phi\sin\xi)^2}$  &     $1/\sqrt{2}$ &   0&       $-1/\sqrt{2}$ &          0 &          0 &          0 &        \\

      &   &  &   $A_2(z)$ &    $|21\rangle$&  $\sqrt{\sin^2(\alpha+\tau)}$ &   0 &      0&    0 &          0 &      0 &          0 &           \\

\hline
\end{tabular}}
\caption{\label{tab: Emission B1g and B2g D4h} Symmetry selectivity in the re-emission process from a ground state $B_{1g}$ or $B_{2g}$ in $D_{4h}$ symmetry. The use of this table is explained in detail in the caption of Tab.~\ref{tab: Emission A1g D4h}.}
\end{table*}


From the absence of spectral weight for $A_{2u}$ contributions to the RIXS intermediate state (see Tabs.~\ref{tab: Absorption B2g D4h} and \ref{tab: Absorption B1g D4h}) we can immediately infer that the $E_g$ states must be suppressed in the final state. However, the $c$-coefficients in $\sigma$ and $\pi$-polarization exchange according to Tab.~\ref{tab: Emission B1g and B2g D4h} and we find

\begin{eqnarray}
|f^\sigma\rangle&=&\frac{1}{\sqrt{2}}\left[ \cos(\tau+\alpha) |11\rangle^+-  |21\rangle^+ - \cos(\tau+\alpha)|31\rangle^+ +  |41\rangle^+\right]\nonumber
\end{eqnarray}

\noindent as well as

\begin{eqnarray}
|f^\pi\rangle&=&\frac{1}{\sqrt{2}}\left[\cos\alpha |11\rangle^++\cos\alpha\cos(\tau+\alpha) |21\rangle^+ + \cos\alpha |31\rangle^+ +\cos\alpha\cos(\tau+\alpha) |41\rangle^+\right]~.\nonumber
\end{eqnarray}

\begin{table*}[htbp]
\resizebox{\columnwidth}{!}{
\begin{tabular}{c|c|c|c|c|c|c|c}
   $\phi$&pol& $A_{1g}$ & A$_{2g}$ & $B_{1g}$ & $B_{2g}$ & $E_g(yz)$ & $E_g(xz)$ \\
    \hline
  0&$\sigma$ & $\frac{1}{2}\cos^2(\alpha+\tau)$ & $\frac{1}{2}$ & $\frac{1}{2}\cos^2(\alpha+\tau)$ & $\frac{1}{2}$ & $0$ & $0$ \\
  0&$\pi$    & $\frac{1}{2}\cos^2\alpha$ & $\frac{1}{2}\cos^2\alpha\cos^2(\alpha+\tau)$ & $\frac{1}{2}\cos^2\alpha$ & $\frac{1}{2}\cos^2\alpha\cos^2(\alpha+\tau)$ & $0$ & $0$ \\
  \hline
\hline
  $\pi/4$&$\sigma$&$\frac{1}{4}(3+\cos\left[2(\alpha+\tau)\right])$ & 0 & 0 & $\frac{1}{4}(3+\cos\left[2(\alpha+\tau)\right])$ &0&0\\
  $\pi/4$&$\pi$& $\frac{1}{4}\cos^2\alpha(3+\cos\left[2(\alpha+\tau)\right])$ & 0 & 0 &$\frac{1}{4}\cos^2\alpha(3+\cos\left[2(\alpha+\tau)\right])$ &0&0\\
\end{tabular}}
\caption{\label{tab: mode spectral weight b2g} Spectral weight of the final states as a function of scattering angle $\tau$ and the incoming light direction $\alpha$ starting from a $B_{1g}$ or $B_{2g}$ groundstate.}
\end{table*}


Clearly, having started from a $B_{1g}$ or $B_{2g}$ ground state flips the $\alpha$-dependence in between $A_{1g}$ and $A_{2g}$ as well as $B_{1g}$ and $B_{2g}$ states with respect to a $A_{1g}$ groundstate. Except of the inverted labeling, Fig.~\ref{fig5} therefore stays valid. Turning the system $\phi=45^{\circ}$ about the $z$ axis gives identical results to the $A_{1g}$ groundstate except for the suppression of $E_g$ states in the former case. Going from $\phi=0$ to $\phi=45^{\circ}$, we thus can expect a complete suppression of states $A_{2g}$ and $B_{1g}$.

It is important however that our estimation only gives information on the \textit{total} symmetry. Assuming that close to zero energy loss, the electronic wave function will hardly have changed. The total symmetry so has to be decomposed into the symmetry of the electronic wave function $B_{1g}$ or $B_{2g}$ and the symmetry of the excitation remaining in the crystal.

It is necessary to note again that formally above method should apply equally for the cuprates. However, previous considerations did not take into account the full symmetry of the wavefunction, which explicitly includes the spin. In order to obtain a reliable result, the ground state symmetry -- including magnetic long range order -- must be known and described by the appropriate magnetic (Shubnikov) point group and \textit{not} by the ordinary point groups. \thefootnote{The coupling coefficients for the magnetic point groups are tabulated in refs.~\onlinecite{Kotzev1980,Kotzev1981,Kotzev1982,Kotzev1982_2} and an application of magnetic point groups to symmetry selection rules is given in Ref.~\onlinecite{Cracknell1967}.} A careful consideration of symmetry selectivity in the antiferromagnetic cuprates has been done in detail in refs.~\onlinecite{vanVeenendaal2006} and \onlinecite{Ament2009}.


\subsection{Intermediate state selection}

In some cases, we find the resonance of a distinct RIXS signal at excitation energies that correspond to a certain orbital configuration of the intermediate or in other words the XAS final state. In anatase TiO$_2$, we e.g. find a phonon resonance at XAS peaks relating to electronic transitions into $d_{x^2-y^2}$ and $d_{z^2}$ orbitals. Such a situation can in principle help to further narrow the criteria for final state selection.
Given this orbital selectivity, the intermediate state must be contained in the representations of the possible core hole state. In Tab.~\ref{tab: intermediate state selection} we apply this idea to a $2p^53d^1$ state and show the multiplication table of the according subspace of $D_{4h}$.

\begin{table*}[htbp]
\begin{center}
\begin{tabular}{c||c|c||c|c}
  ``$O_h$'' & \multicolumn{2}{c||}{``$e_g$''} & \multicolumn{2}{c}{``$t_{2g}$''}  \\
  \hline
  $D_{4h}$ & $B_{1g}(d_{x^2-y^2})$ & $a_{1g}(d_{z^2})$ & $b_{2g}(d_{xy})$ & $e_g(d_{yz}/d_{xz})$  \\
\hline
\hline
  $E_u(p_x/p_y)$ & $E_u$ & $E_u$ & $E_u$ & $A_{1u}$,$A_{2u}$,$B_{1u}$,$B_{2u}$  \\
  $A_{2u}(p_z)$ & $B_{2u}$ & $A_{2u}$ & $B_{1u}$ & $E_u$  \\
\end{tabular}
\caption{\label{tab: intermediate state selection} Intermediate state selection for a $p\rightarrow d$ transition. In a $d^0$ system like TiO$_2$, the excited electron can occupy ``$t_{2g}$'' and ``$e_g$'' states. Thus all direct product intermediate states are possible. In a $d^9$ system like CuO, the excited electron can only occupy the $B_{1g}(x^2-y^2)$ orbital. Intermediate state selection thus allows only the $E_u$ and $B_{2u}$ states of which only the $E_u$ can be reached in XAS.}
\end{center}
\end{table*}

%\begin{table*}[htbp]
%\begin{tabular}{c||c|c||c|c|c}
%  ``$O_h$'' & \multicolumn{2}{c||}{``$e_g$''} & \multicolumn{3}{c}{``$t_{2g}$''}  \\
%  \hline
%  $D_{4h}$ & $B_{1g}(x^2-y^2)$ & $A_{1g}(z^2)$ & $B_{2g}(xy)$ & $E_g(yz)$& $E_g(xz)$  \\
%\hline
%\hline
%  $E_u(x)$ & $E_u$ & $E_u$ & $E_u$ & $A_{2u}$,$B_{1u}$ & $A_{1u}$,$B_{2u}$\\
%  $E_u(y)$ & $E_u$ & $E_u$ & $E_u$ & $A_{1u}$,$B_{2u}$,& $A_{2u}$,$B_{1u}$ \\
%  $A_{2u}(z)$ & 0 & $A_{2u}$ & 0 & $E_u$ & $E_u$ \\
%\end{tabular}
%\label{tab: intermediate state selection}
%\caption{Intermediate state selection.}
%\end{table*}
%

The intermediate states with the electron in a $d_{x^2-y^2}$ or $d_{z^2}$ orbital can contain both the $E_u$ and $A_{2u}$ representations. Therefore, all possible final state symmetries -- discussed in Tab.~\ref{tab: Emission A1g D4h} -- can be expected in the RIXS experiment.

%For RIXS at the oxygen K edge, the situation changes considerably \cite{deGroot2001,deGroot2008}. The ground state configuration is $1s^22s^22p^6$ and the $1s \rightarrow 2p$ channel is formally closed. The O $2p$-states therefore are only visible as result of the covalent bonding of O$2p$ with Ti $3d$ states. Due to negligible Coulomb interaction between core hole and excited electron, multiplet effects are rather weak and the XAS spectrum to a large extent represents the unoccupied density of states of the Ti $3d$ manifold \cite{Asahi,Cheng}. The inset of figure \ref{fig4} shows... The first two peaks are respectively the empty t2g and eg states of the 3d-band and the second structure is related to the titanium 4s- and 4p-bands.  The delocalization effect throughout the entire intermediate regime gives rise to vibronic excitation throughout the entire absorption threshold.
%
%Formally however, the absorption process still needs to consider a O$1s$ to O$2p$ transition or in terms the intermediate state will be formed by one $A_{1g}$ core electron and an electron of $E_u(x,y)$ or $A_{2u}(z)$ symmetry. However, there is an approximately equal amount of oxygen 2p character in the $t_{2g}$ band as in the $e_g$ band which will only be redistributed when the octahedral distortion becomes significant. Thus -- similar to the Ti L-edge, no solid intermediate state selection rule applies. We so expect all phonon modes to be visible at the oxygen K threshold.



%In anatase, we have Raman allowed phonon modes $A_{1g}$, $B_{1g}$ and $E_g$. Our subtraction method discussed before allows us to suppress either modes $A_{1g}$ \textit{or} $B_{1g}$, respectively by going to grazing incidence $\alpha=\pi/2$.




\section{Selfabsorption}

So far, we neglected the angular dependent effect of selfabsorption for the modulation of the RIXS intensity: incident x-rays $h\nu$ traveling a distance $l$ in the solid will exponentially lose intensity

\begin{equation}
I(h\nu,l)\propto e^{-\mu(h\nu) l}~,
\end{equation}

\noindent where $\mu(h\nu)$ is the absorption cross section. Consequently, the quantity $-dI\propto\mu(h\nu) e^{-\mu(h\nu) l}dl$ will be absorbed along the infinitesimal path length $[l,l+dl]$. Some fraction of this number -- e.g. determined by the symmetry considerations discussed in the last chapter -- will cause the RIXS transition of Eq.~\ref{eq: Kramer Heisenberg} and create photons $h\nu'$. On their way $l'(l,\alpha,\tau)$ out of the solid, these x-rays will lose intensity as well, resulting in an effect called self-absorption. According to Fig.~\ref{fig: self absorption geometry}, the law of sines gives

\begin{figure*}[htbp]
\begin{center}
\begin{tikzpicture}

    % define coordinates
    \coordinate (O) at (0,0) ;
    \coordinate (OO) at (0,-2) ;
    \coordinate (A) at (-3,0) ;
    \coordinate (AA) at (-5.25,1.5) ;
    \coordinate (mAA) at (3,-4) ;
    \coordinate (B) at (3,0) ;
    \coordinate (BB) at (4.5,1) ;
    \coordinate (BBB) at (5.25,1.5) ;
    \coordinate (BBBB) at (6,2) ;

    % media
    %\fill[blue!25!,opacity=.3] (-4,0) rectangle (4,2.5);
    \fill[blue!60!,opacity=.3] (-4,0) rectangle (4,-4);
    \node[] at (-1.5,2) {Vacuum};
    \node[] at (-1.5,-3) {Solid};

    % axis
    \draw[dash pattern=on5pt off3pt] (0,-4) -- (0,3) ;
    \draw[dash pattern=on5pt off3pt] (-3,-4) -- (-3,3) ;
    \draw[dash pattern=on5pt off3pt] (3,-4) -- (3,3) ;

    \draw[dash pattern=on5pt off3pt] (-4,-2) -- (4,-2) ;
    \draw[thick] (-3,0) -- (-3,-2) ;
    \node[left] at (-3.2,-1) {$d$};
     \draw[thick] (3,0) -- (3,-2) ;
     \node[right] at (3.2,-1) {$d$};


    % rays
    \draw[red,ultra thick,directed] (A) -- (OO);
      \node[red,right] at (-1.5,-0.75) {$l$};
 \node[red,right] at (-4.5,1.25) {$h\nu$};

    \draw[red,thin] (AA) -- (OO);
     \draw[red,thin,dash pattern=on5pt off3pt] (OO) -- (mAA);
    \draw[blue,thin] (OO) -- (BBB);
    \draw[blue,directed,ultra thick] (OO) -- (B);
 \node[blue,left] at (1.5,-0.75) {$l'$};
 \node[blue,left] at (6,1.25) {$h\nu'$};

    % angles
    \draw (-3,1) arc (90:145.2:1);
    \node[] at (3.9,1.5)  {$\tau+\alpha-\pi$};

    \draw (3,2.5) arc (90:33:2.5) ;
    \node[] at (-3.3,0.5)  {$\alpha$};

    \draw (0.75,-2.5) arc (-34:30:1);
    \node[] at (0.5,-2)  {$\tau$};

\end{tikzpicture}
\caption{\label{fig: self absorption geometry} Self absorption geometry in RIXS.}
\end{center}
\end{figure*}



\begin{equation}
\frac{l}{\cos(\tau+\alpha-\pi)}=\frac{l'}{\cos(\alpha)}~.
\end{equation}

\noindent and we therefore find

\begin{equation}\label{eq: dI RIXS}
dI_{RIXS}(h\nu,h\nu',l)\propto \mu(h\nu) e^{-\left(\mu(h\nu)+\mu(h\nu')\frac{\cos(\alpha)}{\cos(\tau+\alpha-\pi)}\right)l}dl~.
\end{equation}


\subsection{Selfabsorption in a bulk material}

To obtain the total RIXS signal in a bulk material, we need to integrate over the half-space $\int_0^{\infty}dI_{RIXS}$ and obtain

\begin{equation}\label{eq: RIXS self absorption}
I_{RIXS}(h\nu,h\nu')\propto\frac{1}{1+\frac{\mu(h\nu')}{\mu(h\nu)}\frac{\cos(\alpha)}{\cos(\tau+\alpha)}}~.
\end{equation}

Fig.~\ref{fig7} shows the results of Eq.~\ref{eq: RIXS self absorption} for scattering angles $\tau=90^{\circ}$ and $130^{\circ}$ and different ratios $\mu(h\nu')/\mu(h\nu)$ of the absorption coefficient of out and ingoing x-rays.
Clearly, self absorption is not symmetric around the specular angle. Towards grazing incidence, suppression is mostly determined by the absorption of the incident beam. Towards grazing emission however, the traveling distance of the outgoing light is long and suppression of RIXS signal high. Additionally, outgoing photons of different energy are subject to the energy dependence of the absorption cross section $\mu(h\nu')$, which modifies the spectra in an inhomogeneous way.

\begin{figure*}[htbp]
\begin{center}
\includegraphics[width=\columnwidth]{RIXStheory/bulkselfabsorption.pdf}
\caption{\label{fig7} Self absorption in bulk samples. Results are shown for two experimental geometries with \textbf{(left)} $\tau=90^{\circ}$ and \textbf{(right)} $\tau=130^{\circ}$ scattering angle in units of the total RIXS signal created in the solid. The characteristic parameter $\mu(h\nu')/\mu(h\nu)$ is varied over 4 orders of magnitude.}
\end{center}
\end{figure*}

For $\mu(h\nu')\gg \mu(h\nu)$ -- which can be the case when exciting with an energy higher than the main absorption feature in XAS -- significant spectral weight can only be expected close to grazing incidence. If $\mu(h\nu')$ can be neglected however -- as can be the case for RIXS features far away from the absorption edge probed -- signal will only be significantly reduced close to grazing emission. If absorption coefficients of incident light and RIXS signal are comparable, the RIXS intensity to some approximation goes linear for scattering angles $\tau=90^{\circ}$. Towards larger scattering angles $\tau$, this trend flattens considerably and intensity would remain constant for a broad interval of $\alpha$ in backscattering geometry. Experimental configurations with large scattering angle therefore not only provide a wider range of momentum transfer vectors but considerably improve the ``dynamic range'' of angles to work with.



\subsection{Selfabsorption in a thin film}

\begin{figure*}[htbp]
\begin{center}
\includegraphics[width=\columnwidth]{RIXStheory/figure10.pdf}
\caption{\label{fig10} Self absorption in thin film samples. Results are shown for two experimental geometries with \textbf{(left)} $\tau=90^{\circ}$ and \textbf{(right)} $\tau=130^{\circ}$ scattering angle in units of the total RIXS signal created in the solid. The characteristic parameter $\mu(h\nu')/\mu(h\nu)$ is varied over 4 orders of magnitude for three different normalized film thicknesses $d\times\mu(h\nu)$.}
\end{center}
\end{figure*}

In case of a thin film one can neglect the contribution of the bulk and needs to integrate Eq.~\ref{eq: dI RIXS} only over film thickness $d$:

\begin{equation}
\int_0^{d/\cos\alpha}dI_{RIXS}=
\frac{\cos(\alpha+\tau)(e^{\mu(h\nu) \frac{d}{\cos\alpha} \left( \frac{\mu(h\nu')}{\mu(h\nu)} \frac{\cos\alpha}{\cos(\alpha+\tau)} -1\right) }-1)}
{\frac{\mu(h\nu')}{\mu(h\nu)}\cos\alpha-\cos(\alpha+\tau)}
\end{equation}

The results for $\tau=90$ and $130^{\circ}$ scattering geometry obtained for several normalized film thicknesses $d\times\mu(h\nu)=0.1$, 1 and $10$ are shown in Fig.~\ref{fig10}. For small film thicknesses, selfabsorption is quite insensitive to the ratio $\mu(h\nu')/\mu(h\nu)$. The RIXS intensity suppression so remains largely independent on $h\nu'$ and the spectra are modified in a homogeneous way. In the limit of a thick film we approach the bulk result shown in Fig.~\ref{fig7}.


\subsection{$\mu(h\nu)$ and the optical conductivity tensor $\sigma$}

As seen in Sec:~\ref{subsec: Symmetry selectivity in the excitation process}, x-ray absorption and therefore $\mu(h\nu)$ depends on the orientation of the polarization vector $\epsilon$ with respect to the crystal orientation. In titanates like TiO$_2$, dichroism effects are typically weak and $\mu(h\nu)$ thus can be treated isotropically. In the non-isotropic case, the absorption cross section $\mu(h\nu)$ of the incoming x-ray beam becomes a function of the experimental geometry $\alpha$, $\phi$ and the polarization $\xi$. For a fixed scattering geometry, $\mu(h\nu')$ additionally depends on the scattering angle $\tau$. In the following, we show a way how to estimate $\mu(h\nu)$ from simple x-ray absorption experiment preceding the RIXS experiment. We start again from Lambert-Beer's law

\begin{equation}
I(l)=I_0~e^{-\mu\cdot l}~.
\end{equation}


%\noindent where $\mu$ is now a tensor of rank 3 which can be rewritten in terms of the extinction coefficients $\kappa$ such that
%
%\begin{equation}
%\mu=2 k \kappa=\frac{4\pi}{\lambda}\kappa~,
%\end{equation}
%
%\noindent where $k$ is the the wavevector and $\lambda$ the wavelength of light. In linear response theory, $\kappa$ is the imaginary part of the complex refractive index $\tilde{n}=n-i\kappa=\sqrt{\tilde{\epsilon}}$. Thus we can write
%
%\begin{equation}
%\mu=\frac{4\pi}{\lambda}\kappa=\frac{4\pi}{\lambda}\Im{\tilde n}=\frac{2\omega}{c}\Im{\tilde{n}}=\frac{2\omega}{c}\Im{\sqrt{\tilde{\epsilon}}}~,
%\end{equation}
%
%\noindent The dielectric constant $\tilde\epsilon$ can be written in terms of its real part $\epsilon$ and the optical conductivity $\sigma$
%
%\begin{equation}
%\mu=\frac{2\omega}{c}\Im{\sqrt{\tilde{\epsilon}}}=\frac{2\omega}{c}\Im\sqrt{\epsilon+i\frac{\sigma}{\omega}}~,
%\end{equation}
%
%\noindent which for low absorption can be developed as
%
%\begin{equation}
%\mu=\frac{2\omega}{c}\Im\sqrt{\epsilon+i\frac{\sigma}{\omega}}\sim\frac{2\omega}{c}\Im\left(\sqrt{\epsilon}+i\frac{\sigma}{2\sqrt{\epsilon}~\omega}+\mathcal{O}(\sigma^2)\right)=\frac{2\omega}{c}\frac{\sigma}{2\omega\sqrt{\epsilon}}=\frac{\sigma}{c\sqrt{\epsilon}}~.
%\end{equation}
%
%\noindent For x-rays, $\epsilon\sim1$ and we obtain
%
%\begin{equation}
%\mu\sim\frac{\sigma}{c}~.
%\end{equation}

Through the optical theorem, the optical conductivity tensor $\sigma$ is intimately related to x-ray absorption $\mu$, i.e. XAS essentially measures

\begin{equation}\label{eq: optical theorem}
\mu(\phi,\alpha,\xi,\omega)=-\Im\left(\vect{\epsilon}^*(\phi,\alpha,\xi)\cdot\sigma(\omega)\cdot\epsilon(\phi,\alpha,\xi)\right)~,
\end{equation}

\noindent where the polarization vector $\vect{\epsilon}(\phi,\alpha,\xi)$ is defined in Eq.~\ref{eq: polarization vector} and depends on the experimental geometry $\phi$, $\alpha$ and $\xi$. It is therefore always possible to do the XAS measurements in a minimal set of different experimental geometries $\phi$, $\alpha$ and $\xi$, which yields a set of independent linear equations so solve for all components of $\sigma$.


\subsection{Determining $\sigma$ in a non magnetic material }

For non-magnetic materials, the optical conductivity tensor is symmetric and has the form

\begin{equation}
\left(
\begin{array}{ccc}
\sigma _{11}(\omega) & \sigma _{21}(\omega) & \sigma _{31}(\omega) \\
\sigma _{21}(\omega) & \sigma _{22}(\omega) & \sigma _{32}(\omega) \\
\sigma _{31}(\omega) & \sigma _{32}(\omega) & \sigma _{33}(\omega) \\
\end{array}
\right)
\end{equation}

In order to find all 6 independent components of $\sigma$, we need to find 6 different experimental geometries for XAS to obtain a set of 6 independent linear equations \ref{eq: optical theorem}. At a typical beamline, geometries in between normal incidence $\alpha=0$ and $\alpha=\pi/2$ are particularly easy to access. We thus chose arbitrarily for $\alpha=0$ and $\alpha=\pi/4$. The light can typically be selected in $\sigma$- ($\xi=0$) and $\pi$-polarization ($\xi=\pi/2$). Additionally, we can select the azimuth $\phi$ according to the crystal direction with respect to the scattering plane. In these configurations it is easy to find an independent set of linear equations according to Eq.~\ref{eq: optical theorem} to determine the coefficients of $\sigma$ and solely employing linearly polarized light:

\begin{equation}\label{eq: XAS non-magnetic}
\left(
\begin{array}{c}
\mu(0,0,0,\omega)  \\
\mu(0,0,\frac{\pi}{2},\omega)   \\
\mu(0,\frac{\pi}{4},\frac{\pi}{2},\omega)  \\
\mu(\frac{\pi}{4},0,0,\omega)   \\
\mu(\frac{\pi}{2},\frac{\pi}{4},\frac{\pi}{2},\omega)  \\
\mu(\frac{\pi}{4},\frac{\pi}{4},\frac{\pi}{2},\omega)
\end{array}
\right)
=
\left(
\begin{array}{cccccc}
1 & 0 & 0 & 0 & 0 & 0 \\
0 & 0 & 1 & 0 & 0 & 0 \\
0 & 0 & \frac{1}{2} & 0 & 1 & \frac{1}{2} \\
\frac{1}{2} & 1 & \frac{1}{2} & 0 & 0 & 0 \\
\frac{1}{4} & -\frac{1}{2} & \frac{1}{4} & -\frac{1}{\sqrt{2}} & \frac{1}{\sqrt{2}} & \frac{1}{2} \\
\frac{1}{2} & 0 & 0 & -1 & 0 & \frac{1}{2}
\end{array}
\right)\cdot
\left(
\begin{array}{c}
\Im{\sigma _{11}}(\omega)  \\
\Im{\sigma _{21}}(\omega)  \\
\Im{\sigma _{22}}(\omega)   \\
\Im{\sigma _{31}}(\omega)   \\
  \Im{\sigma _{32}}(\omega)  \\
  \Im{\sigma _{33}}(\omega)
\end{array}
\right)
\end{equation}

Inverting this matrix equation, we directly yield the imaginary components of the optical conductivity tensor

\begin{equation}\label{eq: sigma non-magnetic}
\left(
\begin{array}{c}
\Im{\sigma _{11}}(\omega)  \\
\Im{\sigma _{21}}(\omega)   \\
\Im{\sigma _{22}}(\omega)   \\
\Im{\sigma _{31}}(\omega)   \\
  \Im{\sigma _{32}}(\omega)  \\
  \Im{\sigma _{33}}(\omega)
\end{array}
\right)
=
\left(
\begin{array}{cccccc}
1 & 0 & 0 & 0 & 0 & 0 \\
-\frac{1}{2} & -\frac{1}{2} & 0 & 1 & 0 & 0 \\
0 & 1 & 0 & 0 & 0 & 0 \\
\frac{1}{4} \left(2+\sqrt{2}\right) & \frac{1}{2 \sqrt{2}} & 1+\frac{1}{\sqrt{2}} & -\frac{1}{2}-\frac{1}{\sqrt{2}} & -1-\sqrt{2} & \frac{1}{\sqrt{2}} \\
-\frac{1}{2 \sqrt{2}} & \frac{1}{-4+2 \sqrt{2}} & -\frac{1}{\sqrt{2}} & \frac{1}{2}+\frac{1}{\sqrt{2}} & 1+\sqrt{2} & -1-\frac{1}{\sqrt{2}} \\
\frac{1}{\sqrt{2}} & \frac{1}{\sqrt{2}} & 2+\sqrt{2} & -1-\sqrt{2} & -2 \left(1+\sqrt{2}\right) & 2+\sqrt{2} \\
\end{array}
\right)\cdot
\left(
\begin{array}{c}
\mu(0,0,0,\omega)  \\
\mu(0,0,\frac{\pi}{2},\omega)   \\
\mu(0,\frac{\pi}{4},\frac{\pi}{2},\omega)  \\
\mu(\frac{\pi}{4},0,0,\omega)   \\
\mu(\frac{\pi}{2},\frac{\pi}{4},\frac{\pi}{2},\omega)  \\
\mu(\frac{\pi}{4},\frac{\pi}{4},\frac{\pi}{2},\omega)
\end{array}
\right)
\end{equation}


In systems with a higher symmetry than orthorhombic (cubic, tetragonal), the off diagonal components of $\sigma$ are zero $\sigma_{21}=\sigma_{31}=\sigma_{32}=0$ and Eq.~\ref{eq: XAS non-magnetic} simplifies to

\begin{equation}
\left(
\begin{array}{c}
\mu(0,0,0,\omega)  \\
\mu(0,0,\frac{\pi}{2},\omega)   \\
\mu(0,\frac{\pi}{4},\frac{\pi}{2},\omega)
\end{array}
\right)
=
\left(
\begin{array}{ccc}
1  & 0  & 0 \\
0  & 1  & 0 \\
0  & \frac{1}{2} &  \frac{1}{2}
\end{array}
\right)\cdot
\left(
\begin{array}{c}
\Im{\sigma _{11}}(\omega)  \\
\Im{\sigma _{22}}(\omega)   \\
  \Im{\sigma _{33}}(\omega)
\end{array}
\right)~,
\end{equation}

\noindent which is solved by

\begin{equation}
\left(
\begin{array}{c}
\Im{\sigma _{11}}(\omega)  \\
\Im{\sigma _{22}}(\omega)   \\
  \Im{\sigma _{33}}(\omega)
\end{array}
\right)
=
\left(
\begin{array}{cccccc}
1 & 0 & 0 \\
0 & 1 & 0 \\
0 & -1 & 2 \\
\end{array}
\right)\cdot
\left(
\begin{array}{c}
\mu(0,0,0,\omega)  \\
\mu(0,0,\frac{\pi}{2},\omega)   \\
\mu(0,\frac{\pi}{4},\frac{\pi}{2},\omega)
\end{array}
\right)~,
\end{equation}

\noindent and can be even more simplified in a cubic ($\sigma_{11}=\sigma_{22}=\sigma_{33}$) or tetrahedral system ($\sigma_{11}=\sigma_{22}$).

\subsection{Determining $\sigma$ in a magnetic material }

For magnetic materials, the optical conductivity tensor is antisymmetric and has the form

\begin{equation}
\left(
\begin{array}{ccc}
\sigma _{11}(\omega) & -\sigma _{21}(\omega) & -\sigma _{31}(\omega) \\
\sigma _{21}(\omega) & \sigma _{22}(\omega) & -\sigma _{32}(\omega) \\
\sigma _{31}(\omega) & \sigma _{32}(\omega) & \sigma _{33}(\omega) \\
\end{array}
\right)
\end{equation}

In the experimental geometries of our choice $\alpha=0$ and $\alpha=\pi/4$ we now additionally need circular polarized light to find either the real or imaginary part of the optical conductivity $\sigma$. The polarization vector $\vect{\epsilon}(\phi,\alpha,\xi)$ can be now written as

\begin{equation}
\vect{\epsilon}(\phi,\alpha,\pm)=\left(
                  \begin{array}{c}
                    1  \\
                    \pm i\cos\alpha  \\
                    \pm i\sin\alpha  \\
                  \end{array}\right)~.
\end{equation}

Making use of Eq.~\ref{eq: optical theorem}, we find a minimalistic set of 6 independent linear equations

\begin{equation}
\left(
\begin{array}{c}
\mu(0,0,0,\omega)  \\
\mu(0,0,\frac{\pi}{2},\omega)   \\
\mu(0,\frac{\pi}{4},\frac{\pi}{2},\omega)  \\
\mu(0,0,-,\omega)   \\
\mu(0,\frac{\pi}{4},-,\omega)  \\
\mu(\frac{\pi}{2},\frac{\pi}{4},-,\omega)
\end{array}
\right)
=
\left(
\begin{array}{cccccc}
1 & 0 & 0 & 0 & 0 & 0 \\
0 & 1 & 0 & 0 & 0 & 0 \\
0 & \frac{1}{2} & \frac{1}{2} & 0 & 0 & 0 \\
1 & 1 & 0 & 2 & 0 & 0 \\
1 & \frac{1}{2} & \frac{1}{2} & \sqrt{2} & \sqrt{2} & 0 \\
\frac{1}{2} & 1 & \frac{1}{2} & \sqrt{2} & 0 & \sqrt{2} \\
\end{array}
\right)
\left(
\begin{array}{c}
\Im{\sigma _{11}(\omega)}  \\
\Re{\sigma _{21}(\omega)}   \\
\Im{\sigma _{22}(\omega)}   \\
\Re{\sigma _{31}(\omega)}   \\
  \Re{\sigma _{32}(\omega)}  \\
  \Im{\sigma _{33}(\omega)}
\end{array}
\right)~,
\end{equation}

\noindent where ``$+$'' and ``$-$'' indicate left- and righthanded circular polarized light, respectively. Inverting the matrix gives

\begin{equation}
\left(
\begin{array}{c}
\Im{\sigma _{11}(\omega)}  \\
\Re{\sigma _{21}(\omega)}   \\
\Im{\sigma _{22}(\omega)}   \\
\Re{\sigma _{31}(\omega)}   \\
  \Re{\sigma _{32}(\omega)}  \\
  \Im{\sigma _{33}(\omega)}
\end{array}
\right)
=
\left(
\begin{array}{cccccc}
-1 & 0 & 0 & 0 & 0 & 0 \\
\frac{1}{2} & \frac{1}{2} & 0 & -\frac{1}{2} & 0 & 0 \\
0 & -1 & 0 & 0 & 0 & 0 \\
-\frac{1}{2}+\frac{1}{\sqrt{2}} & -\frac{1}{2} & \frac{1}{\sqrt{2}} & \frac{1}{2} & -\frac{1}{\sqrt{2}} & 0 \\
\frac{1}{4} \left(-2+\sqrt{2}\right) & \frac{1}{4} \left(-2+\sqrt{2}\right) & \frac{1}{\sqrt{2}} & \frac{1}{2} & 0 & -\frac{1}{\sqrt{2}} \\
0 & 1 & -2 & 0 & 0 & 0 \\
\end{array}
\right)
\left(
\begin{array}{c}
\mu(0,0,0,\omega)  \\
\mu(0,0,\frac{\pi}{2},\omega)   \\
\mu(0,\frac{\pi}{4},\frac{\pi}{2},\omega)  \\
\mu(0,0,-,\omega)   \\
\mu(0,\frac{\pi}{4},-,\omega)  \\
\mu(\frac{\pi}{2},\frac{\pi}{4},-,\omega)
\end{array}
\right)
\end{equation}


\subsection{Kramers-Kronig Transformation}

In the last paragraph we showed that a simple XAS experiment right yields either the imaginary or the real component of the optical conductivity tensor $\sigma$. Its respective real or imaginary counterpart can be yielded via Kramers-Kronig transformation.

\begin{eqnarray}
\Re\sigma(\omega)&=&\frac{2}{\pi}\dashint_0^\infty \frac{\omega'\Im\sigma(\omega')}{\omega'^2-\omega^2}d\omega'\nonumber\\
\Im\sigma(\omega)&=&-\frac{2\omega}{\pi}\dashint_0^\infty \frac{\Re\sigma(\omega')}{\omega'^2-\omega^2}d\omega'\nonumber
\end{eqnarray}

Unlike in optics, the absorption resonances at the transition metal L-edges are typically reasonably confined in energy space. Further, larger intervals of photon energies are available to yield enough information for a reasonable Kramers-Kronig transformation. Below the L-edge onset, there is typically no structure whereas towards higher energy, some x-ray absorption fine structure can contribute to the signal. An energy range 20~eV below and 50~eV above the absorption threshold thus should give reasonable results.

\subsection{Application of $\sigma$ to the self absorption correction in RIXS }

Knowing $\sigma(\omega)$ we can now calculate $\mu(\phi,\alpha,\xi,h\nu)$ for any geometry of incoming light

\begin{equation}
\mu(\phi,\alpha,\xi,h\nu)\propto-\Im\left(\vect{\epsilon}^*(\phi,\alpha,\xi)\cdot\sigma(h\nu)\cdot\epsilon(\phi,\alpha,\xi)\right)~,
\end{equation}

The polarization plane of the outgoing RIXS signal is fixed by the scattering angle $\tau$ and as seen in Fig.~\ref{fig: ingoing projection}, the argument of $\epsilon$ now needs to be replaced by $\tau+\alpha-\pi$.

\begin{equation}
\mu(\phi,\alpha,\xi',h\nu')\propto-\Im\left(\vect{\epsilon}^*(\phi,\tau+\alpha-\pi,\xi')\cdot\sigma(h\nu')\cdot\epsilon(\phi,\tau+\alpha-\pi,\xi')\right)~,
\end{equation}

The polarization $\xi'$ of the outgoing light however is typically not determined, and has to be typically guessed through symmetry considerations. This will change when the first spectrometers with polarization analyzer, e.g. at the ESRF, will be available. Calculating $\mu(\phi,\alpha,\xi,h\nu)$ and $\mu(\phi,\tau+\alpha-\pi,\xi',h\nu')$ thus allows for a thorough self absorption correction.



\end{widetext}

\begin{thebibliography}{63}
\expandafter\ifx\csname natexlab\endcsname\relax\def\natexlab#1{#1}\fi
\expandafter\ifx\csname bibnamefont\endcsname\relax
  \def\bibnamefont#1{#1}\fi
\expandafter\ifx\csname bibfnamefont\endcsname\relax
  \def\bibfnamefont#1{#1}\fi
\expandafter\ifx\csname citenamefont\endcsname\relax
  \def\citenamefont#1{#1}\fi
\expandafter\ifx\csname url\endcsname\relax
  \def\url#1{\texttt{#1}}\fi
\expandafter\ifx\csname urlprefix\endcsname\relax\def\urlprefix{URL }\fi
\providecommand{\bibinfo}[2]{#2}
\providecommand{\eprint}[2][]{\url{#2}}

%%%%%%%%%%%%%%%%%%%%%%%%%%%%%%%%%
%%%%%%%%%%% RIXS Theory %%%%%%%%%%
%%%%%%%%%%%%%%%%%%%%%%%%%%%%%%%%%

\bibitem{Ament2011}
L.J.P. Ament, M. van Veenendaal, T. P. Devereaux, J. P. Hill and J. van den Brink,
Resonant inelastic x-ray scattering studies of elementary excitations,
Rev. Mod. Phys. {\bf 83}, 705 (2011).

\bibitem{NSLSprojects}
I. Jarrige, D. Arena, A. Baron, Y. Cai, Y.-D. Chuang, F. de Groot, J. Guo,
J.P. Hill, S. Hulbert, C. McGuinness, R. Reininger, J.E. Rubenson, C. Sanchez-Hanke, T. Schmitt and K. Smith,
Soft inelastic x-ray scattering (SIX)
Accepted Proposal, National Synchrotron Light Source II,
Retrieved from http://www.bnl.gov/nsls2/beamlines/2010BeamlineProposal-Approved.asp,
Brookhaven National Laboratory (2010).

\bibitem{slswebpage}
Address user information
Retrieved from http://www.psi.ch/sls/adress/user-information, Swiss Light Source (2014).

\bibitem{Schulke}
W. Sch\"{u}lke,
\textit{Electron dynamics by inelastic x-ray scattering},
Oxford University Press, New York, N.Y., (2007).

\bibitem{Altarelli}
M. Altarelli,
Resonant x-ray scattering: a theoretical introduction,
Lect. Notes Phys. {\bf697}, 201 (2006).

\bibitem{Blume}
M. Blume,
Magnetic scattering of x rays,
J. Appl. Phys. {\bf 57}, 3615 (1985).

\bibitem{Guarise2010}
M. Guarise, B. Dalla Piazza, M. Moretti Sala, G. Ghiringhelli, L. Braicovich, H. Berger, J. N. Hancock, D. van der Marel, T. Schmitt, V. N. Strocov, L. J. P. Ament, J. van den Brink, P.-H. Lin, P. Xu, H. M. R{\o}nnow and M. Grioni,
Measurement of magnetic excitations in the two-dimensional antiferromagnetic Sr$_2$CuO$_2$Cl$_2$ insulator using resonant x-ray scattering: evidence for extended interactions,
Phys. Rev. Lett. {\bf105}, 157006 (2010).

\bibitem{Braicovich2009}
L. Braicovich, L. J. P. Ament, V. Bisogni, F. Forte, C. Aruta, G. Balestrino, N. B. Brookes, G. M. De Luca, P. G. Medaglia, F. Miletto Granozio, M. Radovic, M. Salluzzo, J. van den Brink and G. Ghiringhelli,
Dispersion of magnetic excitations in the cuprate La$_2$CuO$_4$ and CaCuO$_2$ compounds measured using resonant x-ray scattering,
Phys. Rev. Lett {\bf102}, 167401 (2009).

\bibitem{Koster}
G.F. Koster, J. O. Dimmock, R. W. Wheeler and H. Statz,
\textit{Properties of the thirty-two point groups},
M.I.T. Press, Cambridge, M.A. (1963).

\bibitem{Griffith}
J. S. Griffith,
\textit{The irreducible tensor method for molecular symmetry groups},
Englewood Cliffs: Prentice-Hall, N.J. (1962).

\bibitem{Matsubara2000}
M. Matsubara, T. Uozumi, A. Kotani, Y. Harada and S. Shin,
Polarization dependence of resonant x-ray emission spectra in early transition metal compounds,
J. Phys. Soc. Japan {\bf 69}, 1558 (2000).

\bibitem{Nakazawa2000}
M. Nakazawa, H. Ogasawara and A. Kotani,
Theory of polarization dependence in resonant x-ray emission spectroscopy of Ce compounds,
J. Phys. Soc. Japan {\bf69}, 4071 (2000).

\bibitem{Matsubara2002}
M. Matsubara, T. Uozumi, A. Kotani, Y. Harada and S. Shin,
Polarization dependence of resonant x-ray emission spectra in $3d^n$ transition metal compounds with $n=0,~1,~2,~3$,
J. Phys. Soc. Japan {\bf71}, 347 (2002).

\bibitem{Ogasawara2004}
H. Ogasawara, K. Fukui and M. Matsubara,
Polarization dependence of x-ray emission spectroscopy
J. Electron Spectrosc. Relat. Phenom. {\bf136}, 161 (2004).

\bibitem{Tanabe}
Y. Tanabe and S. Sugano,
On the Absorption Spectra of Complex Ions,
J. Physic. Soc. Japan {\bf9}, 753 (1954).

\bibitem{Kamimura}
 Y. Tanabe and H. Kamimura,
On the absorption spectra of complex ions IV. the effect of the spin-orbit interaction and the field of lower symmetry on $d$ electrons in cubic field,
J. Physic. Soc. Japan {\bf13}, 394 (1958).

\bibitem{Fano}
U. Fano , G. Racah,
\textit{Irreducible tensorial sets},
Academic Press, New York, N.Y. (1959).

\bibitem{Snoke}
D. W Snoke,
\textit{32 Point Groups},
Retrieved from http://www.phyast.pitt.edu/~snoke/resources/resources.html (2008).

\bibitem{Kotzev1980}
J. N. Kotzev and M. I. Aroyo,
Clebsch-Gordan coefficients for the corepresentations of Shubnikov point groups,
J. Phys. A: Math. Gen. {\bf13}, 2275 (1980).

\bibitem{Kotzev1981}
J. N. Kotzev and M. I. Aroyo,
Clebsch-Gordan coefficients for the corepresentations of Shubnikov point groups. II. cubic groups,
J. Phys. A: Math. Gen. {\bf14}, 1543 (1981).

\bibitem{Kotzev1982}
J. N. Kotzev and M. I. Aroyo,
Clebsch-Gordan coefficients for the corepresentations of Shubnikov point groups. III. groups of tetragonal, orthorhombic, monoclinic and triclinic crystal systems,
J. Phys. A: Math. Gen. {\bf15}, 711 (1982).

\bibitem{Kotzev1982_2}
J. N. Kotzev and M. I. Aroyo,
Clebsch-Gordan coefficients for the corepresentations of Shubnikov point groups. IV. groups of hexagonal and trigonal systems
J. Phys. A: Math. Gen. {\bf15}, 725 (1982).

\bibitem{Cracknell1967}
A. P. Cracknell and M. R. Daniel,
Magnetic point groups, selection rules and the antiferromagnetic phase transition in UO$_2$,
Proc. Phys. Soc. {\bf92}, 705 (1967).

\bibitem{vanVeenendaal2006}
M. van Veenendaal,
Polarization dependence of $L$- and $M$- edge resonant inelastic x-ray scattering on transition metal compounds,
Phys. Rev. Lett. {\bf96}, 117404 (2006).

\bibitem{Ament2009}
L. J. P. Ament, G. Ghiringhelli, M. M. Sala, L. Braicovich and J. van den Brink
Theoretical demonstration of how the dispersion of magnetic excitations in cuprate compounds can be determined using resonant inelastic x-ray scattering,
Phys. Rev. Lett. {\bf103}, 117003 (2009).

\bibitem{GuarisePhD}
M. Guarise,
Electronic and magnetic resonant inelastic x-ray study of cuprates,
PhD-Thesis, EPFL (2012).







\end{thebibliography}

\end{document}


